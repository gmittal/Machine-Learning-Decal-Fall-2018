
% Default to the notebook output style

    


% Inherit from the specified cell style.




    
\documentclass[11pt]{article}

    
    
    \usepackage[T1]{fontenc}
    % Nicer default font (+ math font) than Computer Modern for most use cases
    \usepackage{mathpazo}

    % Basic figure setup, for now with no caption control since it's done
    % automatically by Pandoc (which extracts ![](path) syntax from Markdown).
    \usepackage{graphicx}
    % We will generate all images so they have a width \maxwidth. This means
    % that they will get their normal width if they fit onto the page, but
    % are scaled down if they would overflow the margins.
    \makeatletter
    \def\maxwidth{\ifdim\Gin@nat@width>\linewidth\linewidth
    \else\Gin@nat@width\fi}
    \makeatother
    \let\Oldincludegraphics\includegraphics
    % Set max figure width to be 80% of text width, for now hardcoded.
    \renewcommand{\includegraphics}[1]{\Oldincludegraphics[width=.8\maxwidth]{#1}}
    % Ensure that by default, figures have no caption (until we provide a
    % proper Figure object with a Caption API and a way to capture that
    % in the conversion process - todo).
    \usepackage{caption}
    \DeclareCaptionLabelFormat{nolabel}{}
    \captionsetup{labelformat=nolabel}

    \usepackage{adjustbox} % Used to constrain images to a maximum size 
    \usepackage{xcolor} % Allow colors to be defined
    \usepackage{enumerate} % Needed for markdown enumerations to work
    \usepackage{geometry} % Used to adjust the document margins
    \usepackage{amsmath} % Equations
    \usepackage{amssymb} % Equations
    \usepackage{textcomp} % defines textquotesingle
    % Hack from http://tex.stackexchange.com/a/47451/13684:
    \AtBeginDocument{%
        \def\PYZsq{\textquotesingle}% Upright quotes in Pygmentized code
    }
    \usepackage{upquote} % Upright quotes for verbatim code
    \usepackage{eurosym} % defines \euro
    \usepackage[mathletters]{ucs} % Extended unicode (utf-8) support
    \usepackage[utf8x]{inputenc} % Allow utf-8 characters in the tex document
    \usepackage{fancyvrb} % verbatim replacement that allows latex
    \usepackage{grffile} % extends the file name processing of package graphics 
                         % to support a larger range 
    % The hyperref package gives us a pdf with properly built
    % internal navigation ('pdf bookmarks' for the table of contents,
    % internal cross-reference links, web links for URLs, etc.)
    \usepackage{hyperref}
    \usepackage{longtable} % longtable support required by pandoc >1.10
    \usepackage{booktabs}  % table support for pandoc > 1.12.2
    \usepackage[inline]{enumitem} % IRkernel/repr support (it uses the enumerate* environment)
    \usepackage[normalem]{ulem} % ulem is needed to support strikethroughs (\sout)
                                % normalem makes italics be italics, not underlines
    

    
    
    % Colors for the hyperref package
    \definecolor{urlcolor}{rgb}{0,.145,.698}
    \definecolor{linkcolor}{rgb}{.71,0.21,0.01}
    \definecolor{citecolor}{rgb}{.12,.54,.11}

    % ANSI colors
    \definecolor{ansi-black}{HTML}{3E424D}
    \definecolor{ansi-black-intense}{HTML}{282C36}
    \definecolor{ansi-red}{HTML}{E75C58}
    \definecolor{ansi-red-intense}{HTML}{B22B31}
    \definecolor{ansi-green}{HTML}{00A250}
    \definecolor{ansi-green-intense}{HTML}{007427}
    \definecolor{ansi-yellow}{HTML}{DDB62B}
    \definecolor{ansi-yellow-intense}{HTML}{B27D12}
    \definecolor{ansi-blue}{HTML}{208FFB}
    \definecolor{ansi-blue-intense}{HTML}{0065CA}
    \definecolor{ansi-magenta}{HTML}{D160C4}
    \definecolor{ansi-magenta-intense}{HTML}{A03196}
    \definecolor{ansi-cyan}{HTML}{60C6C8}
    \definecolor{ansi-cyan-intense}{HTML}{258F8F}
    \definecolor{ansi-white}{HTML}{C5C1B4}
    \definecolor{ansi-white-intense}{HTML}{A1A6B2}

    % commands and environments needed by pandoc snippets
    % extracted from the output of `pandoc -s`
    \providecommand{\tightlist}{%
      \setlength{\itemsep}{0pt}\setlength{\parskip}{0pt}}
    \DefineVerbatimEnvironment{Highlighting}{Verbatim}{commandchars=\\\{\}}
    % Add ',fontsize=\small' for more characters per line
    \newenvironment{Shaded}{}{}
    \newcommand{\KeywordTok}[1]{\textcolor[rgb]{0.00,0.44,0.13}{\textbf{{#1}}}}
    \newcommand{\DataTypeTok}[1]{\textcolor[rgb]{0.56,0.13,0.00}{{#1}}}
    \newcommand{\DecValTok}[1]{\textcolor[rgb]{0.25,0.63,0.44}{{#1}}}
    \newcommand{\BaseNTok}[1]{\textcolor[rgb]{0.25,0.63,0.44}{{#1}}}
    \newcommand{\FloatTok}[1]{\textcolor[rgb]{0.25,0.63,0.44}{{#1}}}
    \newcommand{\CharTok}[1]{\textcolor[rgb]{0.25,0.44,0.63}{{#1}}}
    \newcommand{\StringTok}[1]{\textcolor[rgb]{0.25,0.44,0.63}{{#1}}}
    \newcommand{\CommentTok}[1]{\textcolor[rgb]{0.38,0.63,0.69}{\textit{{#1}}}}
    \newcommand{\OtherTok}[1]{\textcolor[rgb]{0.00,0.44,0.13}{{#1}}}
    \newcommand{\AlertTok}[1]{\textcolor[rgb]{1.00,0.00,0.00}{\textbf{{#1}}}}
    \newcommand{\FunctionTok}[1]{\textcolor[rgb]{0.02,0.16,0.49}{{#1}}}
    \newcommand{\RegionMarkerTok}[1]{{#1}}
    \newcommand{\ErrorTok}[1]{\textcolor[rgb]{1.00,0.00,0.00}{\textbf{{#1}}}}
    \newcommand{\NormalTok}[1]{{#1}}
    
    % Additional commands for more recent versions of Pandoc
    \newcommand{\ConstantTok}[1]{\textcolor[rgb]{0.53,0.00,0.00}{{#1}}}
    \newcommand{\SpecialCharTok}[1]{\textcolor[rgb]{0.25,0.44,0.63}{{#1}}}
    \newcommand{\VerbatimStringTok}[1]{\textcolor[rgb]{0.25,0.44,0.63}{{#1}}}
    \newcommand{\SpecialStringTok}[1]{\textcolor[rgb]{0.73,0.40,0.53}{{#1}}}
    \newcommand{\ImportTok}[1]{{#1}}
    \newcommand{\DocumentationTok}[1]{\textcolor[rgb]{0.73,0.13,0.13}{\textit{{#1}}}}
    \newcommand{\AnnotationTok}[1]{\textcolor[rgb]{0.38,0.63,0.69}{\textbf{\textit{{#1}}}}}
    \newcommand{\CommentVarTok}[1]{\textcolor[rgb]{0.38,0.63,0.69}{\textbf{\textit{{#1}}}}}
    \newcommand{\VariableTok}[1]{\textcolor[rgb]{0.10,0.09,0.49}{{#1}}}
    \newcommand{\ControlFlowTok}[1]{\textcolor[rgb]{0.00,0.44,0.13}{\textbf{{#1}}}}
    \newcommand{\OperatorTok}[1]{\textcolor[rgb]{0.40,0.40,0.40}{{#1}}}
    \newcommand{\BuiltInTok}[1]{{#1}}
    \newcommand{\ExtensionTok}[1]{{#1}}
    \newcommand{\PreprocessorTok}[1]{\textcolor[rgb]{0.74,0.48,0.00}{{#1}}}
    \newcommand{\AttributeTok}[1]{\textcolor[rgb]{0.49,0.56,0.16}{{#1}}}
    \newcommand{\InformationTok}[1]{\textcolor[rgb]{0.38,0.63,0.69}{\textbf{\textit{{#1}}}}}
    \newcommand{\WarningTok}[1]{\textcolor[rgb]{0.38,0.63,0.69}{\textbf{\textit{{#1}}}}}
    
    
    % Define a nice break command that doesn't care if a line doesn't already
    % exist.
    \def\br{\hspace*{\fill} \\* }
    % Math Jax compatability definitions
    \def\gt{>}
    \def\lt{<}
    % Document parameters
    \title{Homework 1}
    
    
    

    % Pygments definitions
    
\makeatletter
\def\PY@reset{\let\PY@it=\relax \let\PY@bf=\relax%
    \let\PY@ul=\relax \let\PY@tc=\relax%
    \let\PY@bc=\relax \let\PY@ff=\relax}
\def\PY@tok#1{\csname PY@tok@#1\endcsname}
\def\PY@toks#1+{\ifx\relax#1\empty\else%
    \PY@tok{#1}\expandafter\PY@toks\fi}
\def\PY@do#1{\PY@bc{\PY@tc{\PY@ul{%
    \PY@it{\PY@bf{\PY@ff{#1}}}}}}}
\def\PY#1#2{\PY@reset\PY@toks#1+\relax+\PY@do{#2}}

\expandafter\def\csname PY@tok@w\endcsname{\def\PY@tc##1{\textcolor[rgb]{0.73,0.73,0.73}{##1}}}
\expandafter\def\csname PY@tok@c\endcsname{\let\PY@it=\textit\def\PY@tc##1{\textcolor[rgb]{0.25,0.50,0.50}{##1}}}
\expandafter\def\csname PY@tok@cp\endcsname{\def\PY@tc##1{\textcolor[rgb]{0.74,0.48,0.00}{##1}}}
\expandafter\def\csname PY@tok@k\endcsname{\let\PY@bf=\textbf\def\PY@tc##1{\textcolor[rgb]{0.00,0.50,0.00}{##1}}}
\expandafter\def\csname PY@tok@kp\endcsname{\def\PY@tc##1{\textcolor[rgb]{0.00,0.50,0.00}{##1}}}
\expandafter\def\csname PY@tok@kt\endcsname{\def\PY@tc##1{\textcolor[rgb]{0.69,0.00,0.25}{##1}}}
\expandafter\def\csname PY@tok@o\endcsname{\def\PY@tc##1{\textcolor[rgb]{0.40,0.40,0.40}{##1}}}
\expandafter\def\csname PY@tok@ow\endcsname{\let\PY@bf=\textbf\def\PY@tc##1{\textcolor[rgb]{0.67,0.13,1.00}{##1}}}
\expandafter\def\csname PY@tok@nb\endcsname{\def\PY@tc##1{\textcolor[rgb]{0.00,0.50,0.00}{##1}}}
\expandafter\def\csname PY@tok@nf\endcsname{\def\PY@tc##1{\textcolor[rgb]{0.00,0.00,1.00}{##1}}}
\expandafter\def\csname PY@tok@nc\endcsname{\let\PY@bf=\textbf\def\PY@tc##1{\textcolor[rgb]{0.00,0.00,1.00}{##1}}}
\expandafter\def\csname PY@tok@nn\endcsname{\let\PY@bf=\textbf\def\PY@tc##1{\textcolor[rgb]{0.00,0.00,1.00}{##1}}}
\expandafter\def\csname PY@tok@ne\endcsname{\let\PY@bf=\textbf\def\PY@tc##1{\textcolor[rgb]{0.82,0.25,0.23}{##1}}}
\expandafter\def\csname PY@tok@nv\endcsname{\def\PY@tc##1{\textcolor[rgb]{0.10,0.09,0.49}{##1}}}
\expandafter\def\csname PY@tok@no\endcsname{\def\PY@tc##1{\textcolor[rgb]{0.53,0.00,0.00}{##1}}}
\expandafter\def\csname PY@tok@nl\endcsname{\def\PY@tc##1{\textcolor[rgb]{0.63,0.63,0.00}{##1}}}
\expandafter\def\csname PY@tok@ni\endcsname{\let\PY@bf=\textbf\def\PY@tc##1{\textcolor[rgb]{0.60,0.60,0.60}{##1}}}
\expandafter\def\csname PY@tok@na\endcsname{\def\PY@tc##1{\textcolor[rgb]{0.49,0.56,0.16}{##1}}}
\expandafter\def\csname PY@tok@nt\endcsname{\let\PY@bf=\textbf\def\PY@tc##1{\textcolor[rgb]{0.00,0.50,0.00}{##1}}}
\expandafter\def\csname PY@tok@nd\endcsname{\def\PY@tc##1{\textcolor[rgb]{0.67,0.13,1.00}{##1}}}
\expandafter\def\csname PY@tok@s\endcsname{\def\PY@tc##1{\textcolor[rgb]{0.73,0.13,0.13}{##1}}}
\expandafter\def\csname PY@tok@sd\endcsname{\let\PY@it=\textit\def\PY@tc##1{\textcolor[rgb]{0.73,0.13,0.13}{##1}}}
\expandafter\def\csname PY@tok@si\endcsname{\let\PY@bf=\textbf\def\PY@tc##1{\textcolor[rgb]{0.73,0.40,0.53}{##1}}}
\expandafter\def\csname PY@tok@se\endcsname{\let\PY@bf=\textbf\def\PY@tc##1{\textcolor[rgb]{0.73,0.40,0.13}{##1}}}
\expandafter\def\csname PY@tok@sr\endcsname{\def\PY@tc##1{\textcolor[rgb]{0.73,0.40,0.53}{##1}}}
\expandafter\def\csname PY@tok@ss\endcsname{\def\PY@tc##1{\textcolor[rgb]{0.10,0.09,0.49}{##1}}}
\expandafter\def\csname PY@tok@sx\endcsname{\def\PY@tc##1{\textcolor[rgb]{0.00,0.50,0.00}{##1}}}
\expandafter\def\csname PY@tok@m\endcsname{\def\PY@tc##1{\textcolor[rgb]{0.40,0.40,0.40}{##1}}}
\expandafter\def\csname PY@tok@gh\endcsname{\let\PY@bf=\textbf\def\PY@tc##1{\textcolor[rgb]{0.00,0.00,0.50}{##1}}}
\expandafter\def\csname PY@tok@gu\endcsname{\let\PY@bf=\textbf\def\PY@tc##1{\textcolor[rgb]{0.50,0.00,0.50}{##1}}}
\expandafter\def\csname PY@tok@gd\endcsname{\def\PY@tc##1{\textcolor[rgb]{0.63,0.00,0.00}{##1}}}
\expandafter\def\csname PY@tok@gi\endcsname{\def\PY@tc##1{\textcolor[rgb]{0.00,0.63,0.00}{##1}}}
\expandafter\def\csname PY@tok@gr\endcsname{\def\PY@tc##1{\textcolor[rgb]{1.00,0.00,0.00}{##1}}}
\expandafter\def\csname PY@tok@ge\endcsname{\let\PY@it=\textit}
\expandafter\def\csname PY@tok@gs\endcsname{\let\PY@bf=\textbf}
\expandafter\def\csname PY@tok@gp\endcsname{\let\PY@bf=\textbf\def\PY@tc##1{\textcolor[rgb]{0.00,0.00,0.50}{##1}}}
\expandafter\def\csname PY@tok@go\endcsname{\def\PY@tc##1{\textcolor[rgb]{0.53,0.53,0.53}{##1}}}
\expandafter\def\csname PY@tok@gt\endcsname{\def\PY@tc##1{\textcolor[rgb]{0.00,0.27,0.87}{##1}}}
\expandafter\def\csname PY@tok@err\endcsname{\def\PY@bc##1{\setlength{\fboxsep}{0pt}\fcolorbox[rgb]{1.00,0.00,0.00}{1,1,1}{\strut ##1}}}
\expandafter\def\csname PY@tok@kc\endcsname{\let\PY@bf=\textbf\def\PY@tc##1{\textcolor[rgb]{0.00,0.50,0.00}{##1}}}
\expandafter\def\csname PY@tok@kd\endcsname{\let\PY@bf=\textbf\def\PY@tc##1{\textcolor[rgb]{0.00,0.50,0.00}{##1}}}
\expandafter\def\csname PY@tok@kn\endcsname{\let\PY@bf=\textbf\def\PY@tc##1{\textcolor[rgb]{0.00,0.50,0.00}{##1}}}
\expandafter\def\csname PY@tok@kr\endcsname{\let\PY@bf=\textbf\def\PY@tc##1{\textcolor[rgb]{0.00,0.50,0.00}{##1}}}
\expandafter\def\csname PY@tok@bp\endcsname{\def\PY@tc##1{\textcolor[rgb]{0.00,0.50,0.00}{##1}}}
\expandafter\def\csname PY@tok@fm\endcsname{\def\PY@tc##1{\textcolor[rgb]{0.00,0.00,1.00}{##1}}}
\expandafter\def\csname PY@tok@vc\endcsname{\def\PY@tc##1{\textcolor[rgb]{0.10,0.09,0.49}{##1}}}
\expandafter\def\csname PY@tok@vg\endcsname{\def\PY@tc##1{\textcolor[rgb]{0.10,0.09,0.49}{##1}}}
\expandafter\def\csname PY@tok@vi\endcsname{\def\PY@tc##1{\textcolor[rgb]{0.10,0.09,0.49}{##1}}}
\expandafter\def\csname PY@tok@vm\endcsname{\def\PY@tc##1{\textcolor[rgb]{0.10,0.09,0.49}{##1}}}
\expandafter\def\csname PY@tok@sa\endcsname{\def\PY@tc##1{\textcolor[rgb]{0.73,0.13,0.13}{##1}}}
\expandafter\def\csname PY@tok@sb\endcsname{\def\PY@tc##1{\textcolor[rgb]{0.73,0.13,0.13}{##1}}}
\expandafter\def\csname PY@tok@sc\endcsname{\def\PY@tc##1{\textcolor[rgb]{0.73,0.13,0.13}{##1}}}
\expandafter\def\csname PY@tok@dl\endcsname{\def\PY@tc##1{\textcolor[rgb]{0.73,0.13,0.13}{##1}}}
\expandafter\def\csname PY@tok@s2\endcsname{\def\PY@tc##1{\textcolor[rgb]{0.73,0.13,0.13}{##1}}}
\expandafter\def\csname PY@tok@sh\endcsname{\def\PY@tc##1{\textcolor[rgb]{0.73,0.13,0.13}{##1}}}
\expandafter\def\csname PY@tok@s1\endcsname{\def\PY@tc##1{\textcolor[rgb]{0.73,0.13,0.13}{##1}}}
\expandafter\def\csname PY@tok@mb\endcsname{\def\PY@tc##1{\textcolor[rgb]{0.40,0.40,0.40}{##1}}}
\expandafter\def\csname PY@tok@mf\endcsname{\def\PY@tc##1{\textcolor[rgb]{0.40,0.40,0.40}{##1}}}
\expandafter\def\csname PY@tok@mh\endcsname{\def\PY@tc##1{\textcolor[rgb]{0.40,0.40,0.40}{##1}}}
\expandafter\def\csname PY@tok@mi\endcsname{\def\PY@tc##1{\textcolor[rgb]{0.40,0.40,0.40}{##1}}}
\expandafter\def\csname PY@tok@il\endcsname{\def\PY@tc##1{\textcolor[rgb]{0.40,0.40,0.40}{##1}}}
\expandafter\def\csname PY@tok@mo\endcsname{\def\PY@tc##1{\textcolor[rgb]{0.40,0.40,0.40}{##1}}}
\expandafter\def\csname PY@tok@ch\endcsname{\let\PY@it=\textit\def\PY@tc##1{\textcolor[rgb]{0.25,0.50,0.50}{##1}}}
\expandafter\def\csname PY@tok@cm\endcsname{\let\PY@it=\textit\def\PY@tc##1{\textcolor[rgb]{0.25,0.50,0.50}{##1}}}
\expandafter\def\csname PY@tok@cpf\endcsname{\let\PY@it=\textit\def\PY@tc##1{\textcolor[rgb]{0.25,0.50,0.50}{##1}}}
\expandafter\def\csname PY@tok@c1\endcsname{\let\PY@it=\textit\def\PY@tc##1{\textcolor[rgb]{0.25,0.50,0.50}{##1}}}
\expandafter\def\csname PY@tok@cs\endcsname{\let\PY@it=\textit\def\PY@tc##1{\textcolor[rgb]{0.25,0.50,0.50}{##1}}}

\def\PYZbs{\char`\\}
\def\PYZus{\char`\_}
\def\PYZob{\char`\{}
\def\PYZcb{\char`\}}
\def\PYZca{\char`\^}
\def\PYZam{\char`\&}
\def\PYZlt{\char`\<}
\def\PYZgt{\char`\>}
\def\PYZsh{\char`\#}
\def\PYZpc{\char`\%}
\def\PYZdl{\char`\$}
\def\PYZhy{\char`\-}
\def\PYZsq{\char`\'}
\def\PYZdq{\char`\"}
\def\PYZti{\char`\~}
% for compatibility with earlier versions
\def\PYZat{@}
\def\PYZlb{[}
\def\PYZrb{]}
\makeatother


    % Exact colors from NB
    \definecolor{incolor}{rgb}{0.0, 0.0, 0.5}
    \definecolor{outcolor}{rgb}{0.545, 0.0, 0.0}



    
    % Prevent overflowing lines due to hard-to-break entities
    \sloppy 
    % Setup hyperref package
    \hypersetup{
      breaklinks=true,  % so long urls are correctly broken across lines
      colorlinks=true,
      urlcolor=urlcolor,
      linkcolor=linkcolor,
      citecolor=citecolor,
      }
    % Slightly bigger margins than the latex defaults
    
    \geometry{verbose,tmargin=1in,bmargin=1in,lmargin=1in,rmargin=1in}
    
    

    \begin{document}
    
    
    \maketitle
    
    

    
    \section{Homework 1}\label{homework-1}

    This homework is intended as a brief overview of the machine learning
process and the various topics you will learn in this class. We hope
that this exercise will allow you to put in context the information you
learn with us this semester. Don't worry if you don't understand the
techniques here (that's what you'll learn this semester!); we just want
to show you how you can use sklearn to do simple machine learning.

    \subsection{Setup}\label{setup}

    First let us import some libraries.

    \begin{Verbatim}[commandchars=\\\{\}]
{\color{incolor}In [{\color{incolor}3}]:} \PY{k+kn}{import} \PY{n+nn}{numpy} \PY{k}{as} \PY{n+nn}{np}
        \PY{k+kn}{import} \PY{n+nn}{matplotlib}\PY{n+nn}{.}\PY{n+nn}{pyplot} \PY{k}{as} \PY{n+nn}{plt}
        \PY{o}{\PYZpc{}}\PY{k}{matplotlib} inline
        
        \PY{k+kn}{from} \PY{n+nn}{sklearn}\PY{n+nn}{.}\PY{n+nn}{datasets} \PY{k}{import} \PY{n}{fetch\PYZus{}mldata}
        \PY{k+kn}{from} \PY{n+nn}{sklearn}\PY{n+nn}{.}\PY{n+nn}{model\PYZus{}selection} \PY{k}{import} \PY{n}{train\PYZus{}test\PYZus{}split}
        \PY{k+kn}{from} \PY{n+nn}{sklearn}\PY{n+nn}{.}\PY{n+nn}{preprocessing} \PY{k}{import} \PY{n}{OneHotEncoder}
        \PY{k+kn}{from} \PY{n+nn}{sklearn}\PY{n+nn}{.}\PY{n+nn}{metrics} \PY{k}{import} \PY{n}{accuracy\PYZus{}score}
        \PY{k+kn}{from} \PY{n+nn}{sklearn}\PY{n+nn}{.}\PY{n+nn}{linear\PYZus{}model} \PY{k}{import} \PY{n}{LinearRegression}\PY{p}{,} \PY{n}{Ridge}\PY{p}{,} \PY{n}{LogisticRegression}
        \PY{k+kn}{from} \PY{n+nn}{sklearn}\PY{n+nn}{.}\PY{n+nn}{ensemble} \PY{k}{import} \PY{n}{RandomForestClassifier}\PY{p}{,} \PY{n}{AdaBoostClassifier}
        \PY{k+kn}{from} \PY{n+nn}{sklearn}\PY{n+nn}{.}\PY{n+nn}{svm} \PY{k}{import} \PY{n}{LinearSVC}\PY{p}{,} \PY{n}{SVC}
        \PY{k+kn}{from} \PY{n+nn}{sklearn}\PY{n+nn}{.}\PY{n+nn}{decomposition} \PY{k}{import} \PY{n}{PCA}
        \PY{k+kn}{from} \PY{n+nn}{sklearn}\PY{n+nn}{.}\PY{n+nn}{neural\PYZus{}network} \PY{k}{import} \PY{n}{MLPClassifier}
\end{Verbatim}


    For this homework assignment, we will be using the MNIST dataset. The
MNIST data is a collection of black and white 28x28 images, each
picturing a handwritten digit. These were collected from digits people
write at the post office, and now this dataset is a standard benchmark
to evaluate models against used in the machine learning community. This
may take some time to download. If this errors out, try rerunning it.

    \begin{Verbatim}[commandchars=\\\{\}]
{\color{incolor}In [{\color{incolor}4}]:} \PY{n}{mnist} \PY{o}{=} \PY{n}{fetch\PYZus{}mldata}\PY{p}{(}\PY{l+s+s1}{\PYZsq{}}\PY{l+s+s1}{MNIST original}\PY{l+s+s1}{\PYZsq{}}\PY{p}{)}
        \PY{n}{X} \PY{o}{=} \PY{n}{mnist}\PY{o}{.}\PY{n}{data}\PY{o}{.}\PY{n}{astype}\PY{p}{(}\PY{l+s+s1}{\PYZsq{}}\PY{l+s+s1}{float64}\PY{l+s+s1}{\PYZsq{}}\PY{p}{)}
        \PY{n}{y} \PY{o}{=} \PY{n}{mnist}\PY{o}{.}\PY{n}{target}\PY{o}{.}\PY{n}{astype}\PY{p}{(}\PY{l+s+s1}{\PYZsq{}}\PY{l+s+s1}{int64}\PY{l+s+s1}{\PYZsq{}}\PY{p}{)}
\end{Verbatim}


    \subsection{Data Exploration}\label{data-exploration}

    Let us first explore this data a little bit.

    \begin{Verbatim}[commandchars=\\\{\}]
{\color{incolor}In [{\color{incolor}5}]:} \PY{n+nb}{print}\PY{p}{(}\PY{n}{X}\PY{o}{.}\PY{n}{shape}\PY{p}{,} \PY{n}{y}\PY{o}{.}\PY{n}{shape}\PY{p}{)} 
\end{Verbatim}


    \begin{Verbatim}[commandchars=\\\{\}]
(70000, 784) (70000,)

    \end{Verbatim}

    The X matrix here contains all the digit pictures. The data is
(n\_samples x n\_features), meaning this data contains 70,000 pictures,
each with 784 features (the 28x28 image is flattened into a single row).
The y vector contains the label for each digit, so we know which digit
(or class - class means category) is in each picture.

    Let's try and visualize this data a bit. Change around the index
variable to explore more.

    \begin{Verbatim}[commandchars=\\\{\}]
{\color{incolor}In [{\color{incolor}6}]:} \PY{n}{index} \PY{o}{=} \PY{l+m+mi}{0} \PY{c+c1}{\PYZsh{}15000, 28999, 67345}
        \PY{n}{image} \PY{o}{=} \PY{n}{X}\PY{p}{[}\PY{n}{index}\PY{p}{]}\PY{o}{.}\PY{n}{reshape}\PY{p}{(}\PY{p}{(}\PY{l+m+mi}{28}\PY{p}{,} \PY{l+m+mi}{28}\PY{p}{)}\PY{p}{)}
        \PY{n}{plt}\PY{o}{.}\PY{n}{title}\PY{p}{(}\PY{l+s+s1}{\PYZsq{}}\PY{l+s+s1}{Label is }\PY{l+s+s1}{\PYZsq{}} \PY{o}{+} \PY{n+nb}{str}\PY{p}{(}\PY{n}{y}\PY{p}{[}\PY{n}{index}\PY{p}{]}\PY{p}{)}\PY{p}{)}
        \PY{n}{plt}\PY{o}{.}\PY{n}{imshow}\PY{p}{(}\PY{n}{image}\PY{p}{,} \PY{n}{cmap}\PY{o}{=}\PY{l+s+s1}{\PYZsq{}}\PY{l+s+s1}{gray}\PY{l+s+s1}{\PYZsq{}}\PY{p}{)}
\end{Verbatim}


\begin{Verbatim}[commandchars=\\\{\}]
{\color{outcolor}Out[{\color{outcolor}6}]:} <matplotlib.image.AxesImage at 0x10b895748>
\end{Verbatim}
            
    \begin{center}
    \adjustimage{max size={0.9\linewidth}{0.9\paperheight}}{output_12_1.png}
    \end{center}
    { \hspace*{\fill} \\}
    
    Notice that each pixel value ranges from 0-255. When we train our
models, a good practice is to \emph{standardize} the data so different
features can be compared more equally. Here we will use a simple
standardization, squeezing all values into the 0-1 interval range.

    \begin{Verbatim}[commandchars=\\\{\}]
{\color{incolor}In [{\color{incolor}7}]:} \PY{n}{X} \PY{o}{=} \PY{n}{X} \PY{o}{/} \PY{l+m+mi}{255}
\end{Verbatim}


    When we train our model, we want it to have the lowest error. Error
presents itself in 2 ways: bias (how close our model is to the ideal
model), and variance (how much our model varies with different
datasets). If we train our model on a chunk of data, and then test our
model on that same data, we will only witness the first type of error -
bias. However, if we test on new, unseen data, that will reflect both
bias and variance. This is the reasoning behind cross validation.

    So, we want to have 2 datasets, train and test, each used for the named
purpose exclusively.

    \begin{Verbatim}[commandchars=\\\{\}]
{\color{incolor}In [{\color{incolor}8}]:} \PY{n}{X\PYZus{}train}\PY{p}{,} \PY{n}{X\PYZus{}test}\PY{p}{,} \PY{n}{y\PYZus{}train}\PY{p}{,} \PY{n}{y\PYZus{}test} \PY{o}{=} \PY{n}{train\PYZus{}test\PYZus{}split}\PY{p}{(}\PY{n}{X}\PY{p}{,} \PY{n}{y}\PY{p}{,} \PY{n}{test\PYZus{}size}\PY{o}{=}\PY{l+m+mf}{0.25}\PY{p}{)}
\end{Verbatim}


    \subsection{Applying Models}\label{applying-models}

    Now we will walk you through applying various models to try and achieve
the lowest error rate on this data.

    Each of our labels is a number from 0-9. If we simply did regression on
this data, the labels would imply some sort of ordering of the classes
(ie the digit 8 is more of the digit 7 than the digit 3 is, etc. We can
fix this issue by one-hot encoding our labels. So, instead of each label
being a simple digit, each label is a vector of 10 entries. 9 of those
entries are zero, and only 1 entry is equal to one, corresponding to the
index of the digit. Let's take a look.

    \begin{Verbatim}[commandchars=\\\{\}]
{\color{incolor}In [{\color{incolor}9}]:} \PY{n}{enc} \PY{o}{=} \PY{n}{OneHotEncoder}\PY{p}{(}\PY{n}{sparse}\PY{o}{=}\PY{k+kc}{False}\PY{p}{)}
        \PY{n}{y\PYZus{}hot} \PY{o}{=} \PY{n}{enc}\PY{o}{.}\PY{n}{fit\PYZus{}transform}\PY{p}{(}\PY{n}{y}\PY{o}{.}\PY{n}{reshape}\PY{p}{(}\PY{o}{\PYZhy{}}\PY{l+m+mi}{1}\PY{p}{,} \PY{l+m+mi}{1}\PY{p}{)}\PY{p}{)}
        \PY{n}{y\PYZus{}train\PYZus{}hot} \PY{o}{=} \PY{n}{enc}\PY{o}{.}\PY{n}{transform}\PY{p}{(}\PY{n}{y\PYZus{}train}\PY{o}{.}\PY{n}{reshape}\PY{p}{(}\PY{o}{\PYZhy{}}\PY{l+m+mi}{1}\PY{p}{,} \PY{l+m+mi}{1}\PY{p}{)}\PY{p}{)}
        \PY{n}{y\PYZus{}hot}\PY{o}{.}\PY{n}{shape}
\end{Verbatim}


\begin{Verbatim}[commandchars=\\\{\}]
{\color{outcolor}Out[{\color{outcolor}9}]:} (70000, 10)
\end{Verbatim}
            
    Remember how the first sample is the digit zero? Let's now look at the
new label at that index.

    \begin{Verbatim}[commandchars=\\\{\}]
{\color{incolor}In [{\color{incolor}10}]:} \PY{n}{y\PYZus{}hot}\PY{p}{[}\PY{l+m+mi}{0}\PY{p}{]}
\end{Verbatim}


\begin{Verbatim}[commandchars=\\\{\}]
{\color{outcolor}Out[{\color{outcolor}10}]:} array([1., 0., 0., 0., 0., 0., 0., 0., 0., 0.])
\end{Verbatim}
            
    \subsubsection{Linear Regression}\label{linear-regression}

    There are 3 steps to build your model: create the model, train the
model, then use your model to make predictions). In the sklearn API,
this is made very clear. First you instantiate the model (constructor),
then you call train it with the \texttt{fit} method, then you can make
predictions on new data with the \texttt{test} method.

    First, let's do a basic linear regression.

    \begin{Verbatim}[commandchars=\\\{\}]
{\color{incolor}In [{\color{incolor}9}]:} \PY{n}{linear} \PY{o}{=} \PY{n}{LinearRegression}\PY{p}{(}\PY{p}{)}
        \PY{n}{linear}\PY{o}{.}\PY{n}{fit}\PY{p}{(}\PY{n}{X\PYZus{}train}\PY{p}{,} \PY{n}{y\PYZus{}train\PYZus{}hot}\PY{p}{)}
\end{Verbatim}


\begin{Verbatim}[commandchars=\\\{\}]
{\color{outcolor}Out[{\color{outcolor}9}]:} LinearRegression(copy\_X=True, fit\_intercept=True, n\_jobs=1, normalize=False)
\end{Verbatim}
            
    \begin{Verbatim}[commandchars=\\\{\}]
{\color{incolor}In [{\color{incolor}10}]:} \PY{c+c1}{\PYZsh{} use trained model to predict both train and test sets}
         \PY{n}{y\PYZus{}train\PYZus{}pred} \PY{o}{=} \PY{n}{linear}\PY{o}{.}\PY{n}{predict}\PY{p}{(}\PY{n}{X\PYZus{}train}\PY{p}{)}
         \PY{n}{y\PYZus{}test\PYZus{}pred} \PY{o}{=} \PY{n}{linear}\PY{o}{.}\PY{n}{predict}\PY{p}{(}\PY{n}{X\PYZus{}test}\PY{p}{)}
         
         \PY{c+c1}{\PYZsh{} print accuracies}
         \PY{n+nb}{print}\PY{p}{(}\PY{l+s+s1}{\PYZsq{}}\PY{l+s+s1}{train acc: }\PY{l+s+s1}{\PYZsq{}}\PY{p}{,} \PY{n}{accuracy\PYZus{}score}\PY{p}{(}\PY{n}{y\PYZus{}train\PYZus{}pred}\PY{o}{.}\PY{n}{argmax}\PY{p}{(}\PY{n}{axis}\PY{o}{=}\PY{l+m+mi}{1}\PY{p}{)}\PY{p}{,} \PY{n}{y\PYZus{}train}\PY{p}{)}\PY{p}{)}
         \PY{n+nb}{print}\PY{p}{(}\PY{l+s+s1}{\PYZsq{}}\PY{l+s+s1}{test acc: }\PY{l+s+s1}{\PYZsq{}}\PY{p}{,} \PY{n}{accuracy\PYZus{}score}\PY{p}{(}\PY{n}{y\PYZus{}test\PYZus{}pred}\PY{o}{.}\PY{n}{argmax}\PY{p}{(}\PY{n}{axis}\PY{o}{=}\PY{l+m+mi}{1}\PY{p}{)}\PY{p}{,} \PY{n}{y\PYZus{}test}\PY{p}{)}\PY{p}{)}
\end{Verbatim}


    \begin{Verbatim}[commandchars=\\\{\}]
train acc:  0.8590666666666666
test acc:  0.8501142857142857

    \end{Verbatim}

    Note on interpretability: you can view the weights of your model with
\texttt{linear.coef\_}

    \subsubsection{Ridge Regression}\label{ridge-regression}

    Let us try and regularize by adding a penalty term to see if we can get
anything better. We can penalize via the L2 norm, aka Ridge Regression.

    \begin{Verbatim}[commandchars=\\\{\}]
{\color{incolor}In [{\color{incolor}11}]:} \PY{n}{ridge} \PY{o}{=} \PY{n}{Ridge}\PY{p}{(}\PY{n}{alpha}\PY{o}{=}\PY{l+m+mf}{0.1}\PY{p}{)}
         \PY{n}{ridge}\PY{o}{.}\PY{n}{fit}\PY{p}{(}\PY{n}{X\PYZus{}train}\PY{p}{,} \PY{n}{y\PYZus{}train\PYZus{}hot}\PY{p}{)}
         \PY{n+nb}{print}\PY{p}{(}\PY{l+s+s1}{\PYZsq{}}\PY{l+s+s1}{train acc: }\PY{l+s+s1}{\PYZsq{}}\PY{p}{,} \PY{n}{accuracy\PYZus{}score}\PY{p}{(}\PY{n}{ridge}\PY{o}{.}\PY{n}{predict}\PY{p}{(}\PY{n}{X\PYZus{}train}\PY{p}{)}\PY{o}{.}\PY{n}{argmax}\PY{p}{(}\PY{n}{axis}\PY{o}{=}\PY{l+m+mi}{1}\PY{p}{)}\PY{p}{,} \PY{n}{y\PYZus{}train}\PY{p}{)}\PY{p}{)}
         \PY{n+nb}{print}\PY{p}{(}\PY{l+s+s1}{\PYZsq{}}\PY{l+s+s1}{test acc: }\PY{l+s+s1}{\PYZsq{}}\PY{p}{,} \PY{n}{accuracy\PYZus{}score}\PY{p}{(}\PY{n}{ridge}\PY{o}{.}\PY{n}{predict}\PY{p}{(}\PY{n}{X\PYZus{}test}\PY{p}{)}\PY{o}{.}\PY{n}{argmax}\PY{p}{(}\PY{n}{axis}\PY{o}{=}\PY{l+m+mi}{1}\PY{p}{)}\PY{p}{,} \PY{n}{y\PYZus{}test}\PY{p}{)}\PY{p}{)}
\end{Verbatim}


    \begin{Verbatim}[commandchars=\\\{\}]
train acc:  0.8589523809523809
test acc:  0.8504571428571429

    \end{Verbatim}

    The alpha controls how much to penalize the weights. Play around with it
to see if you can improve the test accuracy.

    Now you have seen how to use some basic models to fit and evaluate your
data. You will now walk through working with more models. Fill in code
where needed.

    \subsubsection{Logistic Regression}\label{logistic-regression}

    We will now do logistic regression. From now on, the models will
automatically one-hot the labels (so we don't need to worry about it).

    \begin{Verbatim}[commandchars=\\\{\}]
{\color{incolor}In [{\color{incolor}13}]:} \PY{n}{logreg} \PY{o}{=} \PY{n}{LogisticRegression}\PY{p}{(}\PY{n}{C}\PY{o}{=}\PY{l+m+mf}{0.1}\PY{p}{,} \PY{n}{multi\PYZus{}class}\PY{o}{=}\PY{l+s+s1}{\PYZsq{}}\PY{l+s+s1}{multinomial}\PY{l+s+s1}{\PYZsq{}}\PY{p}{,} \PY{n}{solver}\PY{o}{=}\PY{l+s+s1}{\PYZsq{}}\PY{l+s+s1}{saga}\PY{l+s+s1}{\PYZsq{}}\PY{p}{,} \PY{n}{tol}\PY{o}{=}\PY{l+m+mf}{0.1}\PY{p}{)}
         \PY{n}{logreg}\PY{o}{.}\PY{n}{fit}\PY{p}{(}\PY{n}{X\PYZus{}train}\PY{p}{,} \PY{n}{y\PYZus{}train}\PY{p}{)}
         \PY{n+nb}{print}\PY{p}{(}\PY{l+s+s1}{\PYZsq{}}\PY{l+s+s1}{train acc: }\PY{l+s+s1}{\PYZsq{}}\PY{p}{,} \PY{n}{accuracy\PYZus{}score}\PY{p}{(}\PY{n}{logreg}\PY{o}{.}\PY{n}{predict}\PY{p}{(}\PY{n}{X\PYZus{}train}\PY{p}{)}\PY{p}{,} \PY{n}{y\PYZus{}train}\PY{p}{)}\PY{p}{)}
         \PY{n+nb}{print}\PY{p}{(}\PY{l+s+s1}{\PYZsq{}}\PY{l+s+s1}{test acc: }\PY{l+s+s1}{\PYZsq{}}\PY{p}{,} \PY{n}{accuracy\PYZus{}score}\PY{p}{(}\PY{n}{logreg}\PY{o}{.}\PY{n}{predict}\PY{p}{(}\PY{n}{X\PYZus{}test}\PY{p}{)}\PY{p}{,} \PY{n}{y\PYZus{}test}\PY{p}{)}\PY{p}{)}
\end{Verbatim}


    \begin{Verbatim}[commandchars=\\\{\}]
train acc:  0.9280571428571428
test acc:  0.9202857142857143

    \end{Verbatim}

    Our accuracy has jumped \textasciitilde{}5\%! Why is this? Logistic
Regression is a more complex model - instead of computing raw scores as
in linear regression, it does one extra step and squashes values between
0 and 1. This means our model now optimizes over \emph{probabilities}
instead of raw scores. This makes sense since our vectors are 1-hot
encoded.

    The C hyperparameter controls inverse regularization strength (inverse
for this model only). Reguralization is important to make sure our model
doesn't overfit (perform much better on train data than test data). Play
around with the C parameter to try and get better results! You should be
able to hit 92\%.

    \subsubsection{Random Forest}\label{random-forest}

    Decision Trees are a completely different type of classifier. They
essentially break up the possible space by repeatedly "splitting" on
features to keep narrowing down the possibilities. Decision Trees are
normally individually very week, so we typically average them together
in bunches called Random Forest.

    Now you have seen many examples for how to construct, fit, and evaluate
a model. Now do the same for Random Forest using the
\href{http://scikit-learn.org/stable/modules/generated/sklearn.ensemble.RandomForestClassifier.html}{documentation
here}. You should be able to create one easily without needing to
specify any constructor parameters.

    \begin{Verbatim}[commandchars=\\\{\}]
{\color{incolor}In [{\color{incolor}20}]:} \PY{n}{rf} \PY{o}{=} \PY{n}{RandomForestClassifier}\PY{p}{(}\PY{n}{n\PYZus{}estimators}\PY{o}{=}\PY{l+m+mi}{40}\PY{p}{,} \PY{n}{min\PYZus{}samples\PYZus{}split}\PY{o}{=}\PY{l+m+mi}{4}\PY{p}{,} \PY{n}{max\PYZus{}features}\PY{o}{=}\PY{l+m+mi}{100}\PY{p}{)}
         \PY{n}{rf}\PY{o}{.}\PY{n}{fit}\PY{p}{(}\PY{n}{X\PYZus{}train}\PY{p}{,} \PY{n}{y\PYZus{}train}\PY{p}{)}
         \PY{n+nb}{print}\PY{p}{(}\PY{l+s+s1}{\PYZsq{}}\PY{l+s+s1}{train acc: }\PY{l+s+s1}{\PYZsq{}}\PY{p}{,} \PY{n}{accuracy\PYZus{}score}\PY{p}{(}\PY{n}{rf}\PY{o}{.}\PY{n}{predict}\PY{p}{(}\PY{n}{X\PYZus{}train}\PY{p}{)}\PY{p}{,} \PY{n}{y\PYZus{}train}\PY{p}{)}\PY{p}{)}
         \PY{n+nb}{print}\PY{p}{(}\PY{l+s+s1}{\PYZsq{}}\PY{l+s+s1}{test acc: }\PY{l+s+s1}{\PYZsq{}}\PY{p}{,} \PY{n}{accuracy\PYZus{}score}\PY{p}{(}\PY{n}{rf}\PY{o}{.}\PY{n}{predict}\PY{p}{(}\PY{n}{X\PYZus{}test}\PY{p}{)}\PY{p}{,} \PY{n}{y\PYZus{}test}\PY{p}{)}\PY{p}{)}
\end{Verbatim}


    \begin{Verbatim}[commandchars=\\\{\}]
train acc:  0.9998476190476191
test acc:  0.9652571428571428

    \end{Verbatim}

    WOWZA! That train accuracy is amazing, let's see if we can boost up the
test accuracy a bit (since that's what really counts). Try and play
around with the hyperparameters to see if you can edge out more accuracy
(look at the documentation for parameters in the constructor). Focus on
\texttt{n\_estimators}, \texttt{min\_samples\_split},
\texttt{max\_features}. You should be able to hit \textasciitilde{}97\%.

    \subsubsection{SVC}\label{svc}

    A support vector classifier is another completely different type of
classifier. It tries to find the best separating hyperplane through your
data.

    The SVC will toast our laptops unless we reduce the data dimensionality.
Let's keep 80\% of the variation, and get rid of the rest. (This will
cause a slight drop in performance, but not by much).

    \begin{Verbatim}[commandchars=\\\{\}]
{\color{incolor}In [{\color{incolor}11}]:} \PY{n}{pca} \PY{o}{=} \PY{n}{PCA}\PY{p}{(}\PY{n}{n\PYZus{}components}\PY{o}{=}\PY{l+m+mf}{0.8}\PY{p}{,} \PY{n}{whiten}\PY{o}{=}\PY{k+kc}{True}\PY{p}{)}
         \PY{n}{X\PYZus{}train\PYZus{}pca} \PY{o}{=} \PY{n}{pca}\PY{o}{.}\PY{n}{fit\PYZus{}transform}\PY{p}{(}\PY{n}{X\PYZus{}train}\PY{p}{)}
         \PY{n}{X\PYZus{}test\PYZus{}pca} \PY{o}{=} \PY{n}{pca}\PY{o}{.}\PY{n}{transform}\PY{p}{(}\PY{n}{X\PYZus{}test}\PY{p}{)}
\end{Verbatim}


    Great! Now let's take a look at what that actually did.

    \begin{Verbatim}[commandchars=\\\{\}]
{\color{incolor}In [{\color{incolor}22}]:} \PY{n}{X\PYZus{}train\PYZus{}pca}\PY{o}{.}\PY{n}{shape}
\end{Verbatim}


\begin{Verbatim}[commandchars=\\\{\}]
{\color{outcolor}Out[{\color{outcolor}22}]:} (52500, 43)
\end{Verbatim}
            
    Remember, before we had 784 (28x28) features! However, PCA found just 43
basis features that explain 80\% of the data. So, we went to just 5\% of
the original input space, but we still retained 80\% of the information!
Nice.

    This \href{http://colah.github.io/posts/2014-10-Visualizing-MNIST/}{blog
post} explains dimensionality reduction with MNIST far better than I
can. It's a short read (\textless{}10 mins), and it contains some pretty
cool visualizations. Read it and jot down things you learned from the
post or further questions.

    \begin{itemize}
\tightlist
\item
  The manifold hypothesis states that data is in a lower dimensional
  manifold contained within a high-dimensional embedding space
\item
  Through visualizing the components of the data, we can see emerging
  patterns in which data points are clustered together due to
  similarities in the structure of each image embedding (e.g. many 8s
  are together on the graph or many 0s)
\item
  The graph visualization of the data of PCA shows much more structure
  in the clusterings of image data (there is now a clear region where
  most of each digit resides in space)
\item
  Quadratic error functions can allow us to find optimimum dimensions
  using gradient descent (MDS)
\end{itemize}

    Now let's train our first SVC. The LinearSVC can only find a linear
decision boundary (the hyperplane).

    \begin{Verbatim}[commandchars=\\\{\}]
{\color{incolor}In [{\color{incolor}23}]:} \PY{n}{lsvc} \PY{o}{=} \PY{n}{LinearSVC}\PY{p}{(}\PY{n}{dual}\PY{o}{=}\PY{k+kc}{False}\PY{p}{,} \PY{n}{tol}\PY{o}{=}\PY{l+m+mf}{0.01}\PY{p}{)}
         \PY{n}{lsvc}\PY{o}{.}\PY{n}{fit}\PY{p}{(}\PY{n}{X\PYZus{}train\PYZus{}pca}\PY{p}{,} \PY{n}{y\PYZus{}train}\PY{p}{)}
         \PY{n+nb}{print}\PY{p}{(}\PY{l+s+s1}{\PYZsq{}}\PY{l+s+s1}{train acc: }\PY{l+s+s1}{\PYZsq{}}\PY{p}{,} \PY{n}{accuracy\PYZus{}score}\PY{p}{(}\PY{n}{lsvc}\PY{o}{.}\PY{n}{predict}\PY{p}{(}\PY{n}{X\PYZus{}train\PYZus{}pca}\PY{p}{)}\PY{p}{,} \PY{n}{y\PYZus{}train}\PY{p}{)}\PY{p}{)}
         \PY{n+nb}{print}\PY{p}{(}\PY{l+s+s1}{\PYZsq{}}\PY{l+s+s1}{test acc: }\PY{l+s+s1}{\PYZsq{}}\PY{p}{,} \PY{n}{accuracy\PYZus{}score}\PY{p}{(}\PY{n}{lsvc}\PY{o}{.}\PY{n}{predict}\PY{p}{(}\PY{n}{X\PYZus{}test\PYZus{}pca}\PY{p}{)}\PY{p}{,} \PY{n}{y\PYZus{}test}\PY{p}{)}\PY{p}{)}
\end{Verbatim}


    \begin{Verbatim}[commandchars=\\\{\}]
train acc:  0.8926285714285714
test acc:  0.8918285714285714

    \end{Verbatim}

    SVMs are really interesting because they have something called the
\emph{dual formulation}, in which the computation is expressed as
training point inner products. This means that data can be lifted into
higher dimensions easily with this "kernel trick". Data that is not
linearly separable in a lower dimension can be linearly separable in a
higher dimension - which is why we conduct the transform. Let us
experiment.

    A transformation that lifts the data into a higher-dimensional space is
called a kernel. A polynomial kernel expands the feature space by
computing all the polynomial cross terms to a specific degree.

    \begin{Verbatim}[commandchars=\\\{\}]
{\color{incolor}In [{\color{incolor}24}]:} \PY{n}{psvc} \PY{o}{=} \PY{n}{SVC}\PY{p}{(}\PY{n}{kernel}\PY{o}{=}\PY{l+s+s1}{\PYZsq{}}\PY{l+s+s1}{poly}\PY{l+s+s1}{\PYZsq{}}\PY{p}{,} \PY{n}{degree}\PY{o}{=}\PY{l+m+mi}{3}\PY{p}{,} \PY{n}{tol}\PY{o}{=}\PY{l+m+mf}{0.01}\PY{p}{,} \PY{n}{cache\PYZus{}size}\PY{o}{=}\PY{l+m+mi}{4000}\PY{p}{)}
         \PY{n}{psvc}\PY{o}{.}\PY{n}{fit}\PY{p}{(}\PY{n}{X\PYZus{}train\PYZus{}pca}\PY{p}{,} \PY{n}{y\PYZus{}train}\PY{p}{)}
         \PY{n+nb}{print}\PY{p}{(}\PY{l+s+s1}{\PYZsq{}}\PY{l+s+s1}{train acc: }\PY{l+s+s1}{\PYZsq{}}\PY{p}{,} \PY{n}{accuracy\PYZus{}score}\PY{p}{(}\PY{n}{psvc}\PY{o}{.}\PY{n}{predict}\PY{p}{(}\PY{n}{X\PYZus{}train\PYZus{}pca}\PY{p}{)}\PY{p}{,} \PY{n}{y\PYZus{}train}\PY{p}{)}\PY{p}{)}
         \PY{n+nb}{print}\PY{p}{(}\PY{l+s+s1}{\PYZsq{}}\PY{l+s+s1}{test acc: }\PY{l+s+s1}{\PYZsq{}}\PY{p}{,} \PY{n}{accuracy\PYZus{}score}\PY{p}{(}\PY{n}{psvc}\PY{o}{.}\PY{n}{predict}\PY{p}{(}\PY{n}{X\PYZus{}test\PYZus{}pca}\PY{p}{)}\PY{p}{,} \PY{n}{y\PYZus{}test}\PY{p}{)}\PY{p}{)}
\end{Verbatim}


    \begin{Verbatim}[commandchars=\\\{\}]
train acc:  0.9963047619047619
test acc:  0.9805142857142857

    \end{Verbatim}

    Play around with the degree of the polynomial kernel to see what
accuracy you can get.

    The RBF kernel uses the gaussian function to create an infinite
dimensional space - a gaussian peak at each datapoint. Now fiddle with
the \texttt{C} and \texttt{gamma} parameters of the gaussian kernel
below to see what you can get.
\href{http://scikit-learn.org/stable/modules/generated/sklearn.svm.SVC.html}{Here's
documentation}

    \begin{Verbatim}[commandchars=\\\{\}]
{\color{incolor}In [{\color{incolor}25}]:} \PY{n}{rsvc} \PY{o}{=} \PY{n}{SVC}\PY{p}{(}\PY{n}{kernel}\PY{o}{=}\PY{l+s+s1}{\PYZsq{}}\PY{l+s+s1}{rbf}\PY{l+s+s1}{\PYZsq{}}\PY{p}{,} \PY{n}{tol}\PY{o}{=}\PY{l+m+mf}{0.01}\PY{p}{,} \PY{n}{cache\PYZus{}size}\PY{o}{=}\PY{l+m+mi}{4000}\PY{p}{)}
         \PY{n}{rsvc}\PY{o}{.}\PY{n}{fit}\PY{p}{(}\PY{n}{X\PYZus{}train\PYZus{}pca}\PY{p}{,} \PY{n}{y\PYZus{}train}\PY{p}{)}
         \PY{n+nb}{print}\PY{p}{(}\PY{l+s+s1}{\PYZsq{}}\PY{l+s+s1}{train acc: }\PY{l+s+s1}{\PYZsq{}}\PY{p}{,} \PY{n}{accuracy\PYZus{}score}\PY{p}{(}\PY{n}{rsvc}\PY{o}{.}\PY{n}{predict}\PY{p}{(}\PY{n}{X\PYZus{}train\PYZus{}pca}\PY{p}{)}\PY{p}{,} \PY{n}{y\PYZus{}train}\PY{p}{)}\PY{p}{)}
         \PY{n+nb}{print}\PY{p}{(}\PY{l+s+s1}{\PYZsq{}}\PY{l+s+s1}{test acc: }\PY{l+s+s1}{\PYZsq{}}\PY{p}{,} \PY{n}{accuracy\PYZus{}score}\PY{p}{(}\PY{n}{rsvc}\PY{o}{.}\PY{n}{predict}\PY{p}{(}\PY{n}{X\PYZus{}test\PYZus{}pca}\PY{p}{)}\PY{p}{,} \PY{n}{y\PYZus{}test}\PY{p}{)}\PY{p}{)}
\end{Verbatim}


    \begin{Verbatim}[commandchars=\\\{\}]
train acc:  0.9929333333333333
test acc:  0.9826285714285714

    \end{Verbatim}

    Isn't that just amazing accuracy?

    \subsection{Basic Neural Network}\label{basic-neural-network}

    You should never do neural networks in sklearn. Use Keras (which we will
teach you later in this class), Tensorflow, PyTorch, etc. However, in an
effort to keep this homework somewhat cohesive, let us proceed.

    Basic neural networks proceed in layers. Each layer has a certain number
of nodes, representing how expressive that layer can be. Below is a
sample network, with an input layer, one hidden (middle) layer of 50
neurons, and finally the output layer.

    \begin{Verbatim}[commandchars=\\\{\}]
{\color{incolor}In [{\color{incolor}12}]:} \PY{n}{nn} \PY{o}{=} \PY{n}{MLPClassifier}\PY{p}{(}\PY{n}{hidden\PYZus{}layer\PYZus{}sizes}\PY{o}{=}\PY{p}{(}\PY{l+m+mi}{512}\PY{p}{,}\PY{l+m+mi}{512}\PY{p}{,}\PY{l+m+mi}{32}\PY{p}{)}\PY{p}{,} \PY{n}{solver}\PY{o}{=}\PY{l+s+s1}{\PYZsq{}}\PY{l+s+s1}{adam}\PY{l+s+s1}{\PYZsq{}}\PY{p}{,} \PY{n}{verbose}\PY{o}{=}\PY{l+m+mi}{1}\PY{p}{)}
         \PY{n}{nn}\PY{o}{.}\PY{n}{fit}\PY{p}{(}\PY{n}{X\PYZus{}train\PYZus{}pca}\PY{p}{,} \PY{n}{y\PYZus{}train}\PY{p}{)}
         \PY{n+nb}{print}\PY{p}{(}\PY{l+s+s1}{\PYZsq{}}\PY{l+s+s1}{train acc: }\PY{l+s+s1}{\PYZsq{}}\PY{p}{,} \PY{n}{accuracy\PYZus{}score}\PY{p}{(}\PY{n}{nn}\PY{o}{.}\PY{n}{predict}\PY{p}{(}\PY{n}{X\PYZus{}train\PYZus{}pca}\PY{p}{)}\PY{p}{,} \PY{n}{y\PYZus{}train}\PY{p}{)}\PY{p}{)}
         \PY{n+nb}{print}\PY{p}{(}\PY{l+s+s1}{\PYZsq{}}\PY{l+s+s1}{test acc: }\PY{l+s+s1}{\PYZsq{}}\PY{p}{,} \PY{n}{accuracy\PYZus{}score}\PY{p}{(}\PY{n}{nn}\PY{o}{.}\PY{n}{predict}\PY{p}{(}\PY{n}{X\PYZus{}test\PYZus{}pca}\PY{p}{)}\PY{p}{,} \PY{n}{y\PYZus{}test}\PY{p}{)}\PY{p}{)}
\end{Verbatim}


    \begin{Verbatim}[commandchars=\\\{\}]
Iteration 1, loss = 0.30595273
Iteration 2, loss = 0.09344412
Iteration 3, loss = 0.06127277
Iteration 4, loss = 0.04672734
Iteration 5, loss = 0.03160793
Iteration 6, loss = 0.02459260
Iteration 7, loss = 0.02241951
Iteration 8, loss = 0.01877101
Iteration 9, loss = 0.01376453
Iteration 10, loss = 0.01576938
Iteration 11, loss = 0.01467476
Iteration 12, loss = 0.01206055
Iteration 13, loss = 0.00786602
Iteration 14, loss = 0.00659242
Iteration 15, loss = 0.00926799
Iteration 16, loss = 0.01834014
Iteration 17, loss = 0.01338699
Training loss did not improve more than tol=0.000100 for two consecutive epochs. Stopping.
train acc:  0.9975238095238095
test acc:  0.9795428571428572

    \end{Verbatim}

    Fiddle around with the hiddle layers. Change the number of neurons, add
more layers, experiment. You should be able to hit 98\% accuracy.

    Neural networks are optimized with a technique called gradient descent
(a neural net is just one big function - so we can take the gradient
with respect to all its parameters, then just go opposite the gradient
to try and find the minimum). This is why it requires many iterations to
converge.

    \subsection{Turning In}\label{turning-in}

    Convert this notebook to a PDF (file -\textgreater{} download as
-\textgreater{} pdf via latex) and submit to Gradescope.


    % Add a bibliography block to the postdoc
    
    
    
    \end{document}
