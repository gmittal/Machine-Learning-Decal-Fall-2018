
% Default to the notebook output style

    


% Inherit from the specified cell style.




    
\documentclass[11pt]{article}

    
    
    \usepackage[T1]{fontenc}
    % Nicer default font (+ math font) than Computer Modern for most use cases
    \usepackage{mathpazo}

    % Basic figure setup, for now with no caption control since it's done
    % automatically by Pandoc (which extracts ![](path) syntax from Markdown).
    \usepackage{graphicx}
    % We will generate all images so they have a width \maxwidth. This means
    % that they will get their normal width if they fit onto the page, but
    % are scaled down if they would overflow the margins.
    \makeatletter
    \def\maxwidth{\ifdim\Gin@nat@width>\linewidth\linewidth
    \else\Gin@nat@width\fi}
    \makeatother
    \let\Oldincludegraphics\includegraphics
    % Set max figure width to be 80% of text width, for now hardcoded.
    \renewcommand{\includegraphics}[1]{\Oldincludegraphics[width=.8\maxwidth]{#1}}
    % Ensure that by default, figures have no caption (until we provide a
    % proper Figure object with a Caption API and a way to capture that
    % in the conversion process - todo).
    \usepackage{caption}
    \DeclareCaptionLabelFormat{nolabel}{}
    \captionsetup{labelformat=nolabel}

    \usepackage{adjustbox} % Used to constrain images to a maximum size 
    \usepackage{xcolor} % Allow colors to be defined
    \usepackage{enumerate} % Needed for markdown enumerations to work
    \usepackage{geometry} % Used to adjust the document margins
    \usepackage{amsmath} % Equations
    \usepackage{amssymb} % Equations
    \usepackage{textcomp} % defines textquotesingle
    % Hack from http://tex.stackexchange.com/a/47451/13684:
    \AtBeginDocument{%
        \def\PYZsq{\textquotesingle}% Upright quotes in Pygmentized code
    }
    \usepackage{upquote} % Upright quotes for verbatim code
    \usepackage{eurosym} % defines \euro
    \usepackage[mathletters]{ucs} % Extended unicode (utf-8) support
    \usepackage[utf8x]{inputenc} % Allow utf-8 characters in the tex document
    \usepackage{fancyvrb} % verbatim replacement that allows latex
    \usepackage{grffile} % extends the file name processing of package graphics 
                         % to support a larger range 
    % The hyperref package gives us a pdf with properly built
    % internal navigation ('pdf bookmarks' for the table of contents,
    % internal cross-reference links, web links for URLs, etc.)
    \usepackage{hyperref}
    \usepackage{longtable} % longtable support required by pandoc >1.10
    \usepackage{booktabs}  % table support for pandoc > 1.12.2
    \usepackage[inline]{enumitem} % IRkernel/repr support (it uses the enumerate* environment)
    \usepackage[normalem]{ulem} % ulem is needed to support strikethroughs (\sout)
                                % normalem makes italics be italics, not underlines
    

    
    
    % Colors for the hyperref package
    \definecolor{urlcolor}{rgb}{0,.145,.698}
    \definecolor{linkcolor}{rgb}{.71,0.21,0.01}
    \definecolor{citecolor}{rgb}{.12,.54,.11}

    % ANSI colors
    \definecolor{ansi-black}{HTML}{3E424D}
    \definecolor{ansi-black-intense}{HTML}{282C36}
    \definecolor{ansi-red}{HTML}{E75C58}
    \definecolor{ansi-red-intense}{HTML}{B22B31}
    \definecolor{ansi-green}{HTML}{00A250}
    \definecolor{ansi-green-intense}{HTML}{007427}
    \definecolor{ansi-yellow}{HTML}{DDB62B}
    \definecolor{ansi-yellow-intense}{HTML}{B27D12}
    \definecolor{ansi-blue}{HTML}{208FFB}
    \definecolor{ansi-blue-intense}{HTML}{0065CA}
    \definecolor{ansi-magenta}{HTML}{D160C4}
    \definecolor{ansi-magenta-intense}{HTML}{A03196}
    \definecolor{ansi-cyan}{HTML}{60C6C8}
    \definecolor{ansi-cyan-intense}{HTML}{258F8F}
    \definecolor{ansi-white}{HTML}{C5C1B4}
    \definecolor{ansi-white-intense}{HTML}{A1A6B2}

    % commands and environments needed by pandoc snippets
    % extracted from the output of `pandoc -s`
    \providecommand{\tightlist}{%
      \setlength{\itemsep}{0pt}\setlength{\parskip}{0pt}}
    \DefineVerbatimEnvironment{Highlighting}{Verbatim}{commandchars=\\\{\}}
    % Add ',fontsize=\small' for more characters per line
    \newenvironment{Shaded}{}{}
    \newcommand{\KeywordTok}[1]{\textcolor[rgb]{0.00,0.44,0.13}{\textbf{{#1}}}}
    \newcommand{\DataTypeTok}[1]{\textcolor[rgb]{0.56,0.13,0.00}{{#1}}}
    \newcommand{\DecValTok}[1]{\textcolor[rgb]{0.25,0.63,0.44}{{#1}}}
    \newcommand{\BaseNTok}[1]{\textcolor[rgb]{0.25,0.63,0.44}{{#1}}}
    \newcommand{\FloatTok}[1]{\textcolor[rgb]{0.25,0.63,0.44}{{#1}}}
    \newcommand{\CharTok}[1]{\textcolor[rgb]{0.25,0.44,0.63}{{#1}}}
    \newcommand{\StringTok}[1]{\textcolor[rgb]{0.25,0.44,0.63}{{#1}}}
    \newcommand{\CommentTok}[1]{\textcolor[rgb]{0.38,0.63,0.69}{\textit{{#1}}}}
    \newcommand{\OtherTok}[1]{\textcolor[rgb]{0.00,0.44,0.13}{{#1}}}
    \newcommand{\AlertTok}[1]{\textcolor[rgb]{1.00,0.00,0.00}{\textbf{{#1}}}}
    \newcommand{\FunctionTok}[1]{\textcolor[rgb]{0.02,0.16,0.49}{{#1}}}
    \newcommand{\RegionMarkerTok}[1]{{#1}}
    \newcommand{\ErrorTok}[1]{\textcolor[rgb]{1.00,0.00,0.00}{\textbf{{#1}}}}
    \newcommand{\NormalTok}[1]{{#1}}
    
    % Additional commands for more recent versions of Pandoc
    \newcommand{\ConstantTok}[1]{\textcolor[rgb]{0.53,0.00,0.00}{{#1}}}
    \newcommand{\SpecialCharTok}[1]{\textcolor[rgb]{0.25,0.44,0.63}{{#1}}}
    \newcommand{\VerbatimStringTok}[1]{\textcolor[rgb]{0.25,0.44,0.63}{{#1}}}
    \newcommand{\SpecialStringTok}[1]{\textcolor[rgb]{0.73,0.40,0.53}{{#1}}}
    \newcommand{\ImportTok}[1]{{#1}}
    \newcommand{\DocumentationTok}[1]{\textcolor[rgb]{0.73,0.13,0.13}{\textit{{#1}}}}
    \newcommand{\AnnotationTok}[1]{\textcolor[rgb]{0.38,0.63,0.69}{\textbf{\textit{{#1}}}}}
    \newcommand{\CommentVarTok}[1]{\textcolor[rgb]{0.38,0.63,0.69}{\textbf{\textit{{#1}}}}}
    \newcommand{\VariableTok}[1]{\textcolor[rgb]{0.10,0.09,0.49}{{#1}}}
    \newcommand{\ControlFlowTok}[1]{\textcolor[rgb]{0.00,0.44,0.13}{\textbf{{#1}}}}
    \newcommand{\OperatorTok}[1]{\textcolor[rgb]{0.40,0.40,0.40}{{#1}}}
    \newcommand{\BuiltInTok}[1]{{#1}}
    \newcommand{\ExtensionTok}[1]{{#1}}
    \newcommand{\PreprocessorTok}[1]{\textcolor[rgb]{0.74,0.48,0.00}{{#1}}}
    \newcommand{\AttributeTok}[1]{\textcolor[rgb]{0.49,0.56,0.16}{{#1}}}
    \newcommand{\InformationTok}[1]{\textcolor[rgb]{0.38,0.63,0.69}{\textbf{\textit{{#1}}}}}
    \newcommand{\WarningTok}[1]{\textcolor[rgb]{0.38,0.63,0.69}{\textbf{\textit{{#1}}}}}
    
    
    % Define a nice break command that doesn't care if a line doesn't already
    % exist.
    \def\br{\hspace*{\fill} \\* }
    % Math Jax compatability definitions
    \def\gt{>}
    \def\lt{<}
    % Document parameters
    \title{adversarial\_hw}
    
    
    

    % Pygments definitions
    
\makeatletter
\def\PY@reset{\let\PY@it=\relax \let\PY@bf=\relax%
    \let\PY@ul=\relax \let\PY@tc=\relax%
    \let\PY@bc=\relax \let\PY@ff=\relax}
\def\PY@tok#1{\csname PY@tok@#1\endcsname}
\def\PY@toks#1+{\ifx\relax#1\empty\else%
    \PY@tok{#1}\expandafter\PY@toks\fi}
\def\PY@do#1{\PY@bc{\PY@tc{\PY@ul{%
    \PY@it{\PY@bf{\PY@ff{#1}}}}}}}
\def\PY#1#2{\PY@reset\PY@toks#1+\relax+\PY@do{#2}}

\expandafter\def\csname PY@tok@w\endcsname{\def\PY@tc##1{\textcolor[rgb]{0.73,0.73,0.73}{##1}}}
\expandafter\def\csname PY@tok@c\endcsname{\let\PY@it=\textit\def\PY@tc##1{\textcolor[rgb]{0.25,0.50,0.50}{##1}}}
\expandafter\def\csname PY@tok@cp\endcsname{\def\PY@tc##1{\textcolor[rgb]{0.74,0.48,0.00}{##1}}}
\expandafter\def\csname PY@tok@k\endcsname{\let\PY@bf=\textbf\def\PY@tc##1{\textcolor[rgb]{0.00,0.50,0.00}{##1}}}
\expandafter\def\csname PY@tok@kp\endcsname{\def\PY@tc##1{\textcolor[rgb]{0.00,0.50,0.00}{##1}}}
\expandafter\def\csname PY@tok@kt\endcsname{\def\PY@tc##1{\textcolor[rgb]{0.69,0.00,0.25}{##1}}}
\expandafter\def\csname PY@tok@o\endcsname{\def\PY@tc##1{\textcolor[rgb]{0.40,0.40,0.40}{##1}}}
\expandafter\def\csname PY@tok@ow\endcsname{\let\PY@bf=\textbf\def\PY@tc##1{\textcolor[rgb]{0.67,0.13,1.00}{##1}}}
\expandafter\def\csname PY@tok@nb\endcsname{\def\PY@tc##1{\textcolor[rgb]{0.00,0.50,0.00}{##1}}}
\expandafter\def\csname PY@tok@nf\endcsname{\def\PY@tc##1{\textcolor[rgb]{0.00,0.00,1.00}{##1}}}
\expandafter\def\csname PY@tok@nc\endcsname{\let\PY@bf=\textbf\def\PY@tc##1{\textcolor[rgb]{0.00,0.00,1.00}{##1}}}
\expandafter\def\csname PY@tok@nn\endcsname{\let\PY@bf=\textbf\def\PY@tc##1{\textcolor[rgb]{0.00,0.00,1.00}{##1}}}
\expandafter\def\csname PY@tok@ne\endcsname{\let\PY@bf=\textbf\def\PY@tc##1{\textcolor[rgb]{0.82,0.25,0.23}{##1}}}
\expandafter\def\csname PY@tok@nv\endcsname{\def\PY@tc##1{\textcolor[rgb]{0.10,0.09,0.49}{##1}}}
\expandafter\def\csname PY@tok@no\endcsname{\def\PY@tc##1{\textcolor[rgb]{0.53,0.00,0.00}{##1}}}
\expandafter\def\csname PY@tok@nl\endcsname{\def\PY@tc##1{\textcolor[rgb]{0.63,0.63,0.00}{##1}}}
\expandafter\def\csname PY@tok@ni\endcsname{\let\PY@bf=\textbf\def\PY@tc##1{\textcolor[rgb]{0.60,0.60,0.60}{##1}}}
\expandafter\def\csname PY@tok@na\endcsname{\def\PY@tc##1{\textcolor[rgb]{0.49,0.56,0.16}{##1}}}
\expandafter\def\csname PY@tok@nt\endcsname{\let\PY@bf=\textbf\def\PY@tc##1{\textcolor[rgb]{0.00,0.50,0.00}{##1}}}
\expandafter\def\csname PY@tok@nd\endcsname{\def\PY@tc##1{\textcolor[rgb]{0.67,0.13,1.00}{##1}}}
\expandafter\def\csname PY@tok@s\endcsname{\def\PY@tc##1{\textcolor[rgb]{0.73,0.13,0.13}{##1}}}
\expandafter\def\csname PY@tok@sd\endcsname{\let\PY@it=\textit\def\PY@tc##1{\textcolor[rgb]{0.73,0.13,0.13}{##1}}}
\expandafter\def\csname PY@tok@si\endcsname{\let\PY@bf=\textbf\def\PY@tc##1{\textcolor[rgb]{0.73,0.40,0.53}{##1}}}
\expandafter\def\csname PY@tok@se\endcsname{\let\PY@bf=\textbf\def\PY@tc##1{\textcolor[rgb]{0.73,0.40,0.13}{##1}}}
\expandafter\def\csname PY@tok@sr\endcsname{\def\PY@tc##1{\textcolor[rgb]{0.73,0.40,0.53}{##1}}}
\expandafter\def\csname PY@tok@ss\endcsname{\def\PY@tc##1{\textcolor[rgb]{0.10,0.09,0.49}{##1}}}
\expandafter\def\csname PY@tok@sx\endcsname{\def\PY@tc##1{\textcolor[rgb]{0.00,0.50,0.00}{##1}}}
\expandafter\def\csname PY@tok@m\endcsname{\def\PY@tc##1{\textcolor[rgb]{0.40,0.40,0.40}{##1}}}
\expandafter\def\csname PY@tok@gh\endcsname{\let\PY@bf=\textbf\def\PY@tc##1{\textcolor[rgb]{0.00,0.00,0.50}{##1}}}
\expandafter\def\csname PY@tok@gu\endcsname{\let\PY@bf=\textbf\def\PY@tc##1{\textcolor[rgb]{0.50,0.00,0.50}{##1}}}
\expandafter\def\csname PY@tok@gd\endcsname{\def\PY@tc##1{\textcolor[rgb]{0.63,0.00,0.00}{##1}}}
\expandafter\def\csname PY@tok@gi\endcsname{\def\PY@tc##1{\textcolor[rgb]{0.00,0.63,0.00}{##1}}}
\expandafter\def\csname PY@tok@gr\endcsname{\def\PY@tc##1{\textcolor[rgb]{1.00,0.00,0.00}{##1}}}
\expandafter\def\csname PY@tok@ge\endcsname{\let\PY@it=\textit}
\expandafter\def\csname PY@tok@gs\endcsname{\let\PY@bf=\textbf}
\expandafter\def\csname PY@tok@gp\endcsname{\let\PY@bf=\textbf\def\PY@tc##1{\textcolor[rgb]{0.00,0.00,0.50}{##1}}}
\expandafter\def\csname PY@tok@go\endcsname{\def\PY@tc##1{\textcolor[rgb]{0.53,0.53,0.53}{##1}}}
\expandafter\def\csname PY@tok@gt\endcsname{\def\PY@tc##1{\textcolor[rgb]{0.00,0.27,0.87}{##1}}}
\expandafter\def\csname PY@tok@err\endcsname{\def\PY@bc##1{\setlength{\fboxsep}{0pt}\fcolorbox[rgb]{1.00,0.00,0.00}{1,1,1}{\strut ##1}}}
\expandafter\def\csname PY@tok@kc\endcsname{\let\PY@bf=\textbf\def\PY@tc##1{\textcolor[rgb]{0.00,0.50,0.00}{##1}}}
\expandafter\def\csname PY@tok@kd\endcsname{\let\PY@bf=\textbf\def\PY@tc##1{\textcolor[rgb]{0.00,0.50,0.00}{##1}}}
\expandafter\def\csname PY@tok@kn\endcsname{\let\PY@bf=\textbf\def\PY@tc##1{\textcolor[rgb]{0.00,0.50,0.00}{##1}}}
\expandafter\def\csname PY@tok@kr\endcsname{\let\PY@bf=\textbf\def\PY@tc##1{\textcolor[rgb]{0.00,0.50,0.00}{##1}}}
\expandafter\def\csname PY@tok@bp\endcsname{\def\PY@tc##1{\textcolor[rgb]{0.00,0.50,0.00}{##1}}}
\expandafter\def\csname PY@tok@fm\endcsname{\def\PY@tc##1{\textcolor[rgb]{0.00,0.00,1.00}{##1}}}
\expandafter\def\csname PY@tok@vc\endcsname{\def\PY@tc##1{\textcolor[rgb]{0.10,0.09,0.49}{##1}}}
\expandafter\def\csname PY@tok@vg\endcsname{\def\PY@tc##1{\textcolor[rgb]{0.10,0.09,0.49}{##1}}}
\expandafter\def\csname PY@tok@vi\endcsname{\def\PY@tc##1{\textcolor[rgb]{0.10,0.09,0.49}{##1}}}
\expandafter\def\csname PY@tok@vm\endcsname{\def\PY@tc##1{\textcolor[rgb]{0.10,0.09,0.49}{##1}}}
\expandafter\def\csname PY@tok@sa\endcsname{\def\PY@tc##1{\textcolor[rgb]{0.73,0.13,0.13}{##1}}}
\expandafter\def\csname PY@tok@sb\endcsname{\def\PY@tc##1{\textcolor[rgb]{0.73,0.13,0.13}{##1}}}
\expandafter\def\csname PY@tok@sc\endcsname{\def\PY@tc##1{\textcolor[rgb]{0.73,0.13,0.13}{##1}}}
\expandafter\def\csname PY@tok@dl\endcsname{\def\PY@tc##1{\textcolor[rgb]{0.73,0.13,0.13}{##1}}}
\expandafter\def\csname PY@tok@s2\endcsname{\def\PY@tc##1{\textcolor[rgb]{0.73,0.13,0.13}{##1}}}
\expandafter\def\csname PY@tok@sh\endcsname{\def\PY@tc##1{\textcolor[rgb]{0.73,0.13,0.13}{##1}}}
\expandafter\def\csname PY@tok@s1\endcsname{\def\PY@tc##1{\textcolor[rgb]{0.73,0.13,0.13}{##1}}}
\expandafter\def\csname PY@tok@mb\endcsname{\def\PY@tc##1{\textcolor[rgb]{0.40,0.40,0.40}{##1}}}
\expandafter\def\csname PY@tok@mf\endcsname{\def\PY@tc##1{\textcolor[rgb]{0.40,0.40,0.40}{##1}}}
\expandafter\def\csname PY@tok@mh\endcsname{\def\PY@tc##1{\textcolor[rgb]{0.40,0.40,0.40}{##1}}}
\expandafter\def\csname PY@tok@mi\endcsname{\def\PY@tc##1{\textcolor[rgb]{0.40,0.40,0.40}{##1}}}
\expandafter\def\csname PY@tok@il\endcsname{\def\PY@tc##1{\textcolor[rgb]{0.40,0.40,0.40}{##1}}}
\expandafter\def\csname PY@tok@mo\endcsname{\def\PY@tc##1{\textcolor[rgb]{0.40,0.40,0.40}{##1}}}
\expandafter\def\csname PY@tok@ch\endcsname{\let\PY@it=\textit\def\PY@tc##1{\textcolor[rgb]{0.25,0.50,0.50}{##1}}}
\expandafter\def\csname PY@tok@cm\endcsname{\let\PY@it=\textit\def\PY@tc##1{\textcolor[rgb]{0.25,0.50,0.50}{##1}}}
\expandafter\def\csname PY@tok@cpf\endcsname{\let\PY@it=\textit\def\PY@tc##1{\textcolor[rgb]{0.25,0.50,0.50}{##1}}}
\expandafter\def\csname PY@tok@c1\endcsname{\let\PY@it=\textit\def\PY@tc##1{\textcolor[rgb]{0.25,0.50,0.50}{##1}}}
\expandafter\def\csname PY@tok@cs\endcsname{\let\PY@it=\textit\def\PY@tc##1{\textcolor[rgb]{0.25,0.50,0.50}{##1}}}

\def\PYZbs{\char`\\}
\def\PYZus{\char`\_}
\def\PYZob{\char`\{}
\def\PYZcb{\char`\}}
\def\PYZca{\char`\^}
\def\PYZam{\char`\&}
\def\PYZlt{\char`\<}
\def\PYZgt{\char`\>}
\def\PYZsh{\char`\#}
\def\PYZpc{\char`\%}
\def\PYZdl{\char`\$}
\def\PYZhy{\char`\-}
\def\PYZsq{\char`\'}
\def\PYZdq{\char`\"}
\def\PYZti{\char`\~}
% for compatibility with earlier versions
\def\PYZat{@}
\def\PYZlb{[}
\def\PYZrb{]}
\makeatother


    % Exact colors from NB
    \definecolor{incolor}{rgb}{0.0, 0.0, 0.5}
    \definecolor{outcolor}{rgb}{0.545, 0.0, 0.0}



    
    % Prevent overflowing lines due to hard-to-break entities
    \sloppy 
    % Setup hyperref package
    \hypersetup{
      breaklinks=true,  % so long urls are correctly broken across lines
      colorlinks=true,
      urlcolor=urlcolor,
      linkcolor=linkcolor,
      citecolor=citecolor,
      }
    % Slightly bigger margins than the latex defaults
    
    \geometry{verbose,tmargin=1in,bmargin=1in,lmargin=1in,rmargin=1in}
    
    

    \begin{document}
    
    
    \maketitle
    
    

    
    \begin{Verbatim}[commandchars=\\\{\}]
{\color{incolor}In [{\color{incolor}49}]:} \PY{k+kn}{import} \PY{n+nn}{copy}
         \PY{k+kn}{import} \PY{n+nn}{cv2}
         \PY{k+kn}{import} \PY{n+nn}{numpy} \PY{k}{as} \PY{n+nn}{np}
         \PY{k+kn}{import} \PY{n+nn}{os}
         
         \PY{k+kn}{import} \PY{n+nn}{torch}
         \PY{k+kn}{from} \PY{n+nn}{torch}\PY{n+nn}{.}\PY{n+nn}{autograd} \PY{k}{import} \PY{n}{Variable}
         \PY{k+kn}{from} \PY{n+nn}{torchvision} \PY{k}{import} \PY{n}{models}
         \PY{k+kn}{from} \PY{n+nn}{torch} \PY{k}{import} \PY{n}{nn}
         \PY{k+kn}{from} \PY{n+nn}{IPython}\PY{n+nn}{.}\PY{n+nn}{display} \PY{k}{import} \PY{n}{Image}\PY{p}{,} \PY{n}{display}
\end{Verbatim}


    \subsection{Adversarial Examples
Homework}\label{adversarial-examples-homework}

    (Do not be intimidated by the large chunks of code, you do not need to
understand all of it to do the homework. As long as you understand how
adversarial examples work, to fill in the homework you just need to
understand what the comment above the FILL\_IN is about)

    In this assignment, you will learn about adversarial examples.
Adversarial examples are deliberately created inputs that fool a neural
network. For example, in the picture below, by adding noise to an image
of a panda that is correctly classified as a panda, we can fool a neural
network into thinking it is a gibbon.

\begin{figure}
\centering
\includegraphics{./images/panda.png}
\caption{}
\end{figure}

Clearly, this has very dangerous consequences, especially considering
how prevalent computer vision classification systems are. For example,
in the image below physical stickers have been strategically placed on a
stop sign that tricks a self-driving car into thinking it is a different
traffic sign.

\begin{figure}
\centering
\includegraphics{./images/stop_sign.png}
\caption{}
\end{figure}

    Run the cell below, it contains helper functions that will be used. You
do not need to understand the details

    \begin{Verbatim}[commandchars=\\\{\}]
{\color{incolor}In [{\color{incolor}50}]:} \PY{k}{def} \PY{n+nf}{preprocess\PYZus{}image}\PY{p}{(}\PY{n}{cv2im}\PY{p}{,} \PY{n}{resize\PYZus{}im}\PY{o}{=}\PY{k+kc}{True}\PY{p}{)}\PY{p}{:}
             \PY{l+s+sd}{\PYZdq{}\PYZdq{}\PYZdq{}}
         \PY{l+s+sd}{        Processes image for CNNs}
         
         \PY{l+s+sd}{    Args:}
         \PY{l+s+sd}{        PIL\PYZus{}img (PIL\PYZus{}img): Image to process}
         \PY{l+s+sd}{        resize\PYZus{}im (bool): Resize to 224 or not}
         \PY{l+s+sd}{    returns:}
         \PY{l+s+sd}{        im\PYZus{}as\PYZus{}var (Pytorch variable): Variable that contains processed float tensor}
         \PY{l+s+sd}{    \PYZdq{}\PYZdq{}\PYZdq{}}
             \PY{c+c1}{\PYZsh{} mean and std list for channels (Imagenet)}
             \PY{n}{mean} \PY{o}{=} \PY{p}{[}\PY{l+m+mf}{0.485}\PY{p}{,} \PY{l+m+mf}{0.456}\PY{p}{,} \PY{l+m+mf}{0.406}\PY{p}{]}
             \PY{n}{std} \PY{o}{=} \PY{p}{[}\PY{l+m+mf}{0.229}\PY{p}{,} \PY{l+m+mf}{0.224}\PY{p}{,} \PY{l+m+mf}{0.225}\PY{p}{]}
             \PY{c+c1}{\PYZsh{} Resize image}
             \PY{k}{if} \PY{n}{resize\PYZus{}im}\PY{p}{:}
                 \PY{n}{cv2im} \PY{o}{=} \PY{n}{cv2}\PY{o}{.}\PY{n}{resize}\PY{p}{(}\PY{n}{cv2im}\PY{p}{,} \PY{p}{(}\PY{l+m+mi}{224}\PY{p}{,} \PY{l+m+mi}{224}\PY{p}{)}\PY{p}{)}
             \PY{n}{im\PYZus{}as\PYZus{}arr} \PY{o}{=} \PY{n}{np}\PY{o}{.}\PY{n}{float32}\PY{p}{(}\PY{n}{cv2im}\PY{p}{)}
             \PY{n}{im\PYZus{}as\PYZus{}arr} \PY{o}{=} \PY{n}{np}\PY{o}{.}\PY{n}{ascontiguousarray}\PY{p}{(}\PY{n}{im\PYZus{}as\PYZus{}arr}\PY{p}{[}\PY{o}{.}\PY{o}{.}\PY{o}{.}\PY{p}{,} \PY{p}{:}\PY{p}{:}\PY{o}{\PYZhy{}}\PY{l+m+mi}{1}\PY{p}{]}\PY{p}{)}
             \PY{n}{im\PYZus{}as\PYZus{}arr} \PY{o}{=} \PY{n}{im\PYZus{}as\PYZus{}arr}\PY{o}{.}\PY{n}{transpose}\PY{p}{(}\PY{l+m+mi}{2}\PY{p}{,} \PY{l+m+mi}{0}\PY{p}{,} \PY{l+m+mi}{1}\PY{p}{)}  \PY{c+c1}{\PYZsh{} Convert array to D,W,H}
             \PY{c+c1}{\PYZsh{} Normalize the channels}
             \PY{k}{for} \PY{n}{channel}\PY{p}{,} \PY{n}{\PYZus{}} \PY{o+ow}{in} \PY{n+nb}{enumerate}\PY{p}{(}\PY{n}{im\PYZus{}as\PYZus{}arr}\PY{p}{)}\PY{p}{:}
                 \PY{n}{im\PYZus{}as\PYZus{}arr}\PY{p}{[}\PY{n}{channel}\PY{p}{]} \PY{o}{/}\PY{o}{=} \PY{l+m+mi}{255}
                 \PY{n}{im\PYZus{}as\PYZus{}arr}\PY{p}{[}\PY{n}{channel}\PY{p}{]} \PY{o}{\PYZhy{}}\PY{o}{=} \PY{n}{mean}\PY{p}{[}\PY{n}{channel}\PY{p}{]}
                 \PY{n}{im\PYZus{}as\PYZus{}arr}\PY{p}{[}\PY{n}{channel}\PY{p}{]} \PY{o}{/}\PY{o}{=} \PY{n}{std}\PY{p}{[}\PY{n}{channel}\PY{p}{]}
             \PY{c+c1}{\PYZsh{} Convert to float tensor}
             \PY{n}{im\PYZus{}as\PYZus{}ten} \PY{o}{=} \PY{n}{torch}\PY{o}{.}\PY{n}{from\PYZus{}numpy}\PY{p}{(}\PY{n}{im\PYZus{}as\PYZus{}arr}\PY{p}{)}\PY{o}{.}\PY{n}{float}\PY{p}{(}\PY{p}{)}
             \PY{c+c1}{\PYZsh{} Add one more channel to the beginning. Tensor shape = 1,3,224,224}
             \PY{n}{im\PYZus{}as\PYZus{}ten}\PY{o}{.}\PY{n}{unsqueeze\PYZus{}}\PY{p}{(}\PY{l+m+mi}{0}\PY{p}{)}
             \PY{c+c1}{\PYZsh{} Convert to Pytorch variable}
             \PY{n}{im\PYZus{}as\PYZus{}var} \PY{o}{=} \PY{n}{Variable}\PY{p}{(}\PY{n}{im\PYZus{}as\PYZus{}ten}\PY{p}{,} \PY{n}{requires\PYZus{}grad}\PY{o}{=}\PY{k+kc}{True}\PY{p}{)}
             \PY{k}{return} \PY{n}{im\PYZus{}as\PYZus{}var}
         
         
         \PY{k}{def} \PY{n+nf}{recreate\PYZus{}image}\PY{p}{(}\PY{n}{im\PYZus{}as\PYZus{}var}\PY{p}{)}\PY{p}{:}
             \PY{l+s+sd}{\PYZdq{}\PYZdq{}\PYZdq{}}
         \PY{l+s+sd}{        Recreates images from a torch variable, sort of reverse preprocessing}
         
         \PY{l+s+sd}{    Args:}
         \PY{l+s+sd}{        im\PYZus{}as\PYZus{}var (torch variable): Image to recreate}
         
         \PY{l+s+sd}{    returns:}
         \PY{l+s+sd}{        recreated\PYZus{}im (numpy arr): Recreated image in array}
         \PY{l+s+sd}{    \PYZdq{}\PYZdq{}\PYZdq{}}
             \PY{n}{reverse\PYZus{}mean} \PY{o}{=} \PY{p}{[}\PY{o}{\PYZhy{}}\PY{l+m+mf}{0.485}\PY{p}{,} \PY{o}{\PYZhy{}}\PY{l+m+mf}{0.456}\PY{p}{,} \PY{o}{\PYZhy{}}\PY{l+m+mf}{0.406}\PY{p}{]}
             \PY{n}{reverse\PYZus{}std} \PY{o}{=} \PY{p}{[}\PY{l+m+mi}{1}\PY{o}{/}\PY{l+m+mf}{0.229}\PY{p}{,} \PY{l+m+mi}{1}\PY{o}{/}\PY{l+m+mf}{0.224}\PY{p}{,} \PY{l+m+mi}{1}\PY{o}{/}\PY{l+m+mf}{0.225}\PY{p}{]}
             \PY{n}{recreated\PYZus{}im} \PY{o}{=} \PY{n}{copy}\PY{o}{.}\PY{n}{copy}\PY{p}{(}\PY{n}{im\PYZus{}as\PYZus{}var}\PY{o}{.}\PY{n}{data}\PY{o}{.}\PY{n}{numpy}\PY{p}{(}\PY{p}{)}\PY{p}{[}\PY{l+m+mi}{0}\PY{p}{]}\PY{p}{)}
             \PY{k}{for} \PY{n}{c} \PY{o+ow}{in} \PY{n+nb}{range}\PY{p}{(}\PY{l+m+mi}{3}\PY{p}{)}\PY{p}{:}
                 \PY{n}{recreated\PYZus{}im}\PY{p}{[}\PY{n}{c}\PY{p}{]} \PY{o}{/}\PY{o}{=} \PY{n}{reverse\PYZus{}std}\PY{p}{[}\PY{n}{c}\PY{p}{]}
                 \PY{n}{recreated\PYZus{}im}\PY{p}{[}\PY{n}{c}\PY{p}{]} \PY{o}{\PYZhy{}}\PY{o}{=} \PY{n}{reverse\PYZus{}mean}\PY{p}{[}\PY{n}{c}\PY{p}{]}
             \PY{n}{recreated\PYZus{}im}\PY{p}{[}\PY{n}{recreated\PYZus{}im} \PY{o}{\PYZgt{}} \PY{l+m+mi}{1}\PY{p}{]} \PY{o}{=} \PY{l+m+mi}{1}
             \PY{n}{recreated\PYZus{}im}\PY{p}{[}\PY{n}{recreated\PYZus{}im} \PY{o}{\PYZlt{}} \PY{l+m+mi}{0}\PY{p}{]} \PY{o}{=} \PY{l+m+mi}{0}
             \PY{n}{recreated\PYZus{}im} \PY{o}{=} \PY{n}{np}\PY{o}{.}\PY{n}{round}\PY{p}{(}\PY{n}{recreated\PYZus{}im} \PY{o}{*} \PY{l+m+mi}{255}\PY{p}{)}
         
             \PY{n}{recreated\PYZus{}im} \PY{o}{=} \PY{n}{np}\PY{o}{.}\PY{n}{uint8}\PY{p}{(}\PY{n}{recreated\PYZus{}im}\PY{p}{)}\PY{o}{.}\PY{n}{transpose}\PY{p}{(}\PY{l+m+mi}{1}\PY{p}{,} \PY{l+m+mi}{2}\PY{p}{,} \PY{l+m+mi}{0}\PY{p}{)}
             \PY{c+c1}{\PYZsh{} Convert RBG to GBR}
             \PY{n}{recreated\PYZus{}im} \PY{o}{=} \PY{n}{recreated\PYZus{}im}\PY{p}{[}\PY{o}{.}\PY{o}{.}\PY{o}{.}\PY{p}{,} \PY{p}{:}\PY{p}{:}\PY{o}{\PYZhy{}}\PY{l+m+mi}{1}\PY{p}{]}
             \PY{k}{return} \PY{n}{recreated\PYZus{}im}
         
         
         \PY{k}{def} \PY{n+nf}{get\PYZus{}params}\PY{p}{(}\PY{n}{example\PYZus{}index}\PY{p}{)}\PY{p}{:}
             \PY{l+s+sd}{\PYZdq{}\PYZdq{}\PYZdq{}}
         \PY{l+s+sd}{        Gets used variables for almost all visualizations, like the image, model etc.}
         
         \PY{l+s+sd}{    Args:}
         \PY{l+s+sd}{        example\PYZus{}index (int): Image id to use from examples}
         
         \PY{l+s+sd}{    returns:}
         \PY{l+s+sd}{        original\PYZus{}image (numpy arr): Original image read from the file}
         \PY{l+s+sd}{        prep\PYZus{}img (numpy\PYZus{}arr): Processed image}
         \PY{l+s+sd}{        target\PYZus{}class (int): Target class for the image}
         \PY{l+s+sd}{        file\PYZus{}name\PYZus{}to\PYZus{}export (string): File name to export the visualizations}
         \PY{l+s+sd}{        pretrained\PYZus{}model(Pytorch model): Model to use for the operations}
         \PY{l+s+sd}{    \PYZdq{}\PYZdq{}\PYZdq{}}
             \PY{c+c1}{\PYZsh{} Pick one of the examples}
             \PY{n}{example\PYZus{}list} \PY{o}{=} \PY{p}{[}\PY{p}{[}\PY{l+s+s1}{\PYZsq{}}\PY{l+s+s1}{./input\PYZus{}images/apple.JPEG}\PY{l+s+s1}{\PYZsq{}}\PY{p}{,} \PY{l+m+mi}{948}\PY{p}{]}\PY{p}{,}
                             \PY{p}{[}\PY{l+s+s1}{\PYZsq{}}\PY{l+s+s1}{./input\PYZus{}images/eel.JPEG}\PY{l+s+s1}{\PYZsq{}}\PY{p}{,} \PY{l+m+mi}{390}\PY{p}{]}\PY{p}{,}
                             \PY{p}{[}\PY{l+s+s1}{\PYZsq{}}\PY{l+s+s1}{./input\PYZus{}images/bird.JPEG}\PY{l+s+s1}{\PYZsq{}}\PY{p}{,} \PY{l+m+mi}{13}\PY{p}{]}\PY{p}{]}
             \PY{n}{selected\PYZus{}example} \PY{o}{=} \PY{n}{example\PYZus{}index}
             \PY{n}{img\PYZus{}path} \PY{o}{=} \PY{n}{example\PYZus{}list}\PY{p}{[}\PY{n}{selected\PYZus{}example}\PY{p}{]}\PY{p}{[}\PY{l+m+mi}{0}\PY{p}{]}
             \PY{n}{target\PYZus{}class} \PY{o}{=} \PY{n}{example\PYZus{}list}\PY{p}{[}\PY{n}{selected\PYZus{}example}\PY{p}{]}\PY{p}{[}\PY{l+m+mi}{1}\PY{p}{]}
             \PY{n}{file\PYZus{}name\PYZus{}to\PYZus{}export} \PY{o}{=} \PY{n}{img\PYZus{}path}\PY{p}{[}\PY{n}{img\PYZus{}path}\PY{o}{.}\PY{n}{rfind}\PY{p}{(}\PY{l+s+s1}{\PYZsq{}}\PY{l+s+s1}{/}\PY{l+s+s1}{\PYZsq{}}\PY{p}{)}\PY{o}{+}\PY{l+m+mi}{1}\PY{p}{:}\PY{n}{img\PYZus{}path}\PY{o}{.}\PY{n}{rfind}\PY{p}{(}\PY{l+s+s1}{\PYZsq{}}\PY{l+s+s1}{.}\PY{l+s+s1}{\PYZsq{}}\PY{p}{)}\PY{p}{]}
             \PY{c+c1}{\PYZsh{} Read image}
             \PY{n}{original\PYZus{}image} \PY{o}{=} \PY{n}{cv2}\PY{o}{.}\PY{n}{imread}\PY{p}{(}\PY{n}{img\PYZus{}path}\PY{p}{,} \PY{l+m+mi}{1}\PY{p}{)}
             \PY{c+c1}{\PYZsh{} Process image}
             \PY{n}{prep\PYZus{}img} \PY{o}{=} \PY{n}{preprocess\PYZus{}image}\PY{p}{(}\PY{n}{original\PYZus{}image}\PY{p}{)}
             \PY{c+c1}{\PYZsh{} Define model}
             \PY{n}{pretrained\PYZus{}model} \PY{o}{=} \PY{n}{models}\PY{o}{.}\PY{n}{alexnet}\PY{p}{(}\PY{n}{pretrained}\PY{o}{=}\PY{k+kc}{True}\PY{p}{)}
             \PY{k}{return} \PY{p}{(}\PY{n}{img\PYZus{}path}\PY{p}{,} \PY{n}{original\PYZus{}image}\PY{p}{,}
                     \PY{n}{prep\PYZus{}img}\PY{p}{,}
                     \PY{n}{target\PYZus{}class}\PY{p}{,}
                     \PY{n}{file\PYZus{}name\PYZus{}to\PYZus{}export}\PY{p}{,}
                     \PY{n}{pretrained\PYZus{}model}\PY{p}{)}
\end{Verbatim}


    Today we will explore the Fast Gradient Sign Method (FGSM) of generating
adversarial examples. You can find the paper that introduces it at this
link if you are interested in reading more about it:
https://arxiv.org/abs/1412.6572

The idea behind this method is in a sense the reversal of
backpropagation. This is because we want to calculate the gradient of
the cost (same as backprop so far) with respect to the input image
pixels (different since with respect to model weights for backprop). In
a sense, we want to know how perturbing each pixel will affect the cost,
which will in turn affect what label the machine learning model
classifies an image as. Once we know this, we know exactly how to
exploit and perturb the image the minimal amount in order to get the
model to classify it as a different label. Mathematically, FGSM takes
the gradient computed from the description above, and converts each
number into either +1 or -1 depending on its sign. Then, it multiplies
this by a very small epsilon value and adds it to the original image.
This new resulting image will then be classified incorrectly by the
model this attack was created against. Pretty terrifying how easy it is
to fool these models, huh.

The loss function used in this attack is the cross entropy loss between
what the model predicts, and the actual label used. To get the label the
model predicts that the loss is run on, you will need to actually call
the model on the input image. The result of this is what is used for the
loss function described.

    \begin{Verbatim}[commandchars=\\\{\}]
{\color{incolor}In [{\color{incolor}51}]:} \PY{c+c1}{\PYZsh{} Fast Gradient Sign Untargeted to Fill In }
         
         \PY{k}{class} \PY{n+nc}{FastGradientSignUntargeted}\PY{p}{(}\PY{p}{)}\PY{p}{:}
             \PY{l+s+sd}{\PYZdq{}\PYZdq{}\PYZdq{}}
         \PY{l+s+sd}{        Fast gradient sign untargeted adversarial attack, minimizes the initial class activation}
         \PY{l+s+sd}{        with iterative grad sign updates}
         \PY{l+s+sd}{    \PYZdq{}\PYZdq{}\PYZdq{}}
             \PY{k}{def} \PY{n+nf}{\PYZus{}\PYZus{}init\PYZus{}\PYZus{}}\PY{p}{(}\PY{n+nb+bp}{self}\PY{p}{,} \PY{n}{model}\PY{p}{,} \PY{n}{alpha}\PY{p}{)}\PY{p}{:}
                 \PY{n+nb+bp}{self}\PY{o}{.}\PY{n}{model} \PY{o}{=} \PY{n}{model}
                 \PY{n+nb+bp}{self}\PY{o}{.}\PY{n}{model}\PY{o}{.}\PY{n}{eval}\PY{p}{(}\PY{p}{)}
                 \PY{c+c1}{\PYZsh{} Movement multiplier per iteration}
                 \PY{n+nb+bp}{self}\PY{o}{.}\PY{n}{alpha} \PY{o}{=} \PY{n}{alpha}
                 \PY{c+c1}{\PYZsh{} Create the folder to export images if not exists}
                 \PY{k}{if} \PY{o+ow}{not} \PY{n}{os}\PY{o}{.}\PY{n}{path}\PY{o}{.}\PY{n}{exists}\PY{p}{(}\PY{l+s+s1}{\PYZsq{}}\PY{l+s+s1}{./generated}\PY{l+s+s1}{\PYZsq{}}\PY{p}{)}\PY{p}{:}
                     \PY{n}{os}\PY{o}{.}\PY{n}{makedirs}\PY{p}{(}\PY{l+s+s1}{\PYZsq{}}\PY{l+s+s1}{./generated}\PY{l+s+s1}{\PYZsq{}}\PY{p}{)}
         
             \PY{k}{def} \PY{n+nf}{generate}\PY{p}{(}\PY{n+nb+bp}{self}\PY{p}{,} \PY{n}{original\PYZus{}image}\PY{p}{,} \PY{n}{im\PYZus{}label}\PY{p}{)}\PY{p}{:}
                 \PY{c+c1}{\PYZsh{} image label as variable}
                 \PY{n}{im\PYZus{}label\PYZus{}as\PYZus{}var} \PY{o}{=} \PY{n}{Variable}\PY{p}{(}\PY{n}{torch}\PY{o}{.}\PY{n}{from\PYZus{}numpy}\PY{p}{(}\PY{n}{np}\PY{o}{.}\PY{n}{asarray}\PY{p}{(}\PY{p}{[}\PY{n}{im\PYZus{}label}\PY{p}{]}\PY{p}{)}\PY{p}{)}\PY{p}{)}
                 \PY{c+c1}{\PYZsh{} Define loss functions}
                 \PY{n}{ce\PYZus{}loss} \PY{o}{=} \PY{n}{nn}\PY{o}{.}\PY{n}{CrossEntropyLoss}\PY{p}{(}\PY{p}{)}
                 \PY{c+c1}{\PYZsh{} Process image}
                 \PY{n}{processed\PYZus{}image} \PY{o}{=} \PY{n}{preprocess\PYZus{}image}\PY{p}{(}\PY{n}{original\PYZus{}image}\PY{p}{)}
                 \PY{c+c1}{\PYZsh{} Start iteration}
                 \PY{k}{for} \PY{n}{i} \PY{o+ow}{in} \PY{n+nb}{range}\PY{p}{(}\PY{l+m+mi}{10}\PY{p}{)}\PY{p}{:}
                     \PY{n+nb}{print}\PY{p}{(}\PY{l+s+s1}{\PYZsq{}}\PY{l+s+s1}{Iteration:}\PY{l+s+s1}{\PYZsq{}}\PY{p}{,} \PY{n+nb}{str}\PY{p}{(}\PY{n}{i}\PY{p}{)}\PY{p}{)}
                     \PY{c+c1}{\PYZsh{} zero\PYZus{}gradients(x)}
                     \PY{c+c1}{\PYZsh{} Zero out previous gradients}
                     \PY{c+c1}{\PYZsh{} Can also use zero\PYZus{}gradients(x)}
                     \PY{n}{processed\PYZus{}image}\PY{o}{.}\PY{n}{grad} \PY{o}{=} \PY{k+kc}{None}
                     \PY{c+c1}{\PYZsh{} Forward pass}
                     \PY{n}{model\PYZus{}output} \PY{o}{=} \PY{n+nb+bp}{self}\PY{o}{.}\PY{n}{model}\PY{p}{(}\PY{n}{processed\PYZus{}image}\PY{p}{)}
                     \PY{c+c1}{\PYZsh{} Calculate CE loss}
                     \PY{n}{pred\PYZus{}loss} \PY{o}{=} \PY{n}{ce\PYZus{}loss}\PY{p}{(}\PY{n}{model\PYZus{}output}\PY{p}{,} \PY{n}{im\PYZus{}label\PYZus{}as\PYZus{}var}\PY{p}{)}
                     \PY{c+c1}{\PYZsh{} Do backward pass}
                     \PY{n}{pred\PYZus{}loss}\PY{o}{.}\PY{n}{backward}\PY{p}{(}\PY{p}{)}
                     \PY{c+c1}{\PYZsh{} Create Noise}
                     \PY{c+c1}{\PYZsh{} Here, processed\PYZus{}image.grad.data is also the same thing is the backward gradient from}
                     \PY{c+c1}{\PYZsh{} the first layer, can use that with hooks as well}
                     \PY{n}{adv\PYZus{}noise} \PY{o}{=} \PY{n+nb+bp}{self}\PY{o}{.}\PY{n}{alpha} \PY{o}{*} \PY{n}{torch}\PY{o}{.}\PY{n}{sign}\PY{p}{(}\PY{n}{processed\PYZus{}image}\PY{o}{.}\PY{n}{grad}\PY{o}{.}\PY{n}{data}\PY{p}{)}
                     \PY{c+c1}{\PYZsh{} Add Noise to processed image}
                     \PY{n}{processed\PYZus{}image}\PY{o}{.}\PY{n}{data} \PY{o}{=} \PY{n}{processed\PYZus{}image} \PY{o}{+} \PY{n}{adv\PYZus{}noise}
         
                     \PY{c+c1}{\PYZsh{} Confirming if the image is indeed adversarial with added noise}
                     \PY{c+c1}{\PYZsh{} This is necessary (for some cases) because when we recreate image}
                     \PY{c+c1}{\PYZsh{} the values become integers between 1 and 255 and sometimes the adversariality}
                     \PY{c+c1}{\PYZsh{} is lost in the recreation process}
         
                     \PY{c+c1}{\PYZsh{} Generate confirmation image}
                     \PY{n}{recreated\PYZus{}image} \PY{o}{=} \PY{n}{recreate\PYZus{}image}\PY{p}{(}\PY{n}{processed\PYZus{}image}\PY{p}{)}
                     \PY{c+c1}{\PYZsh{} Process confirmation image}
                     \PY{n}{prep\PYZus{}confirmation\PYZus{}image} \PY{o}{=} \PY{n}{preprocess\PYZus{}image}\PY{p}{(}\PY{n}{recreated\PYZus{}image}\PY{p}{)}
                     \PY{c+c1}{\PYZsh{} Forward pass to make sure creating the adversarial example was successful}
                     \PY{n}{confirmation\PYZus{}out} \PY{o}{=} \PY{n+nb+bp}{self}\PY{o}{.}\PY{n}{model}\PY{p}{(}\PY{n}{prep\PYZus{}confirmation\PYZus{}image}\PY{p}{)}
                     \PY{c+c1}{\PYZsh{} Get prediction}
                     \PY{n}{\PYZus{}}\PY{p}{,} \PY{n}{confirmation\PYZus{}prediction} \PY{o}{=} \PY{n}{confirmation\PYZus{}out}\PY{o}{.}\PY{n}{data}\PY{o}{.}\PY{n}{max}\PY{p}{(}\PY{l+m+mi}{1}\PY{p}{)}
                     \PY{c+c1}{\PYZsh{} Get Probability}
                     \PY{n}{confirmation\PYZus{}confidence} \PY{o}{=} \PYZbs{}
                         \PY{n}{nn}\PY{o}{.}\PY{n}{functional}\PY{o}{.}\PY{n}{softmax}\PY{p}{(}\PY{n}{confirmation\PYZus{}out}\PY{p}{)}\PY{p}{[}\PY{l+m+mi}{0}\PY{p}{]}\PY{p}{[}\PY{n}{confirmation\PYZus{}prediction}\PY{p}{]}\PY{o}{.}\PY{n}{data}\PY{o}{.}\PY{n}{numpy}\PY{p}{(}\PY{p}{)}\PY{p}{[}\PY{l+m+mi}{0}\PY{p}{]}
                     \PY{c+c1}{\PYZsh{} Convert tensor to int}
                     \PY{n}{confirmation\PYZus{}prediction} \PY{o}{=} \PY{n}{confirmation\PYZus{}prediction}\PY{o}{.}\PY{n}{numpy}\PY{p}{(}\PY{p}{)}\PY{p}{[}\PY{l+m+mi}{0}\PY{p}{]}
                     \PY{c+c1}{\PYZsh{} Check if the prediction is different than the original}
                     \PY{k}{if} \PY{n}{confirmation\PYZus{}prediction} \PY{o}{!=} \PY{n}{im\PYZus{}label}\PY{p}{:}
                         \PY{n+nb}{print}\PY{p}{(}\PY{l+s+s1}{\PYZsq{}}\PY{l+s+s1}{Original image was predicted as:}\PY{l+s+s1}{\PYZsq{}}\PY{p}{,} \PY{n}{im\PYZus{}label}\PY{p}{,}
                               \PY{l+s+s1}{\PYZsq{}}\PY{l+s+s1}{with adversarial noise converted to:}\PY{l+s+s1}{\PYZsq{}}\PY{p}{,} \PY{n}{confirmation\PYZus{}prediction}\PY{p}{,}
                               \PY{l+s+s1}{\PYZsq{}}\PY{l+s+s1}{and predicted with confidence of:}\PY{l+s+s1}{\PYZsq{}}\PY{p}{,} \PY{n}{confirmation\PYZus{}confidence}\PY{p}{)}
                         \PY{c+c1}{\PYZsh{} Create the image for noise, which is the difference between the}
                         \PY{c+c1}{\PYZsh{} adversarial example and original image}
                         \PY{n}{noise\PYZus{}image} \PY{o}{=} \PY{n}{confirmation\PYZus{}prediction} \PY{o}{\PYZhy{}} \PY{n}{original\PYZus{}image}
                         \PY{n}{name\PYZus{}noise} \PY{o}{=} \PY{l+s+s1}{\PYZsq{}}\PY{l+s+s1}{./generated/untargeted\PYZus{}adv\PYZus{}noise\PYZus{}from\PYZus{}}\PY{l+s+s1}{\PYZsq{}} \PY{o}{+} \PY{n+nb}{str}\PY{p}{(}\PY{n}{im\PYZus{}label}\PY{p}{)} \PY{o}{+} \PY{l+s+s1}{\PYZsq{}}\PY{l+s+s1}{\PYZus{}to\PYZus{}}\PY{l+s+s1}{\PYZsq{}} \PY{o}{+} \PY{n+nb}{str}\PY{p}{(}\PY{n}{confirmation\PYZus{}prediction}\PY{p}{)} \PY{o}{+} \PY{l+s+s1}{\PYZsq{}}\PY{l+s+s1}{.jpg}\PY{l+s+s1}{\PYZsq{}}
                         \PY{n}{cv2}\PY{o}{.}\PY{n}{imwrite}\PY{p}{(}\PY{n}{name\PYZus{}noise}\PY{p}{,} \PY{n}{noise\PYZus{}image}\PY{p}{)}
                         \PY{c+c1}{\PYZsh{} Write image}
                         \PY{n}{name\PYZus{}image} \PY{o}{=} \PY{l+s+s1}{\PYZsq{}}\PY{l+s+s1}{./generated/untargeted\PYZus{}adv\PYZus{}img\PYZus{}from\PYZus{}}\PY{l+s+s1}{\PYZsq{}} \PY{o}{+} \PY{n+nb}{str}\PY{p}{(}\PY{n}{im\PYZus{}label}\PY{p}{)} \PY{o}{+} \PY{l+s+s1}{\PYZsq{}}\PY{l+s+s1}{\PYZus{}to\PYZus{}}\PY{l+s+s1}{\PYZsq{}} \PY{o}{+} \PY{n+nb}{str}\PY{p}{(}\PY{n}{confirmation\PYZus{}prediction}\PY{p}{)} \PY{o}{+} \PY{l+s+s1}{\PYZsq{}}\PY{l+s+s1}{.jpg}\PY{l+s+s1}{\PYZsq{}}
                         \PY{n}{cv2}\PY{o}{.}\PY{n}{imwrite}\PY{p}{(}\PY{n}{name\PYZus{}image}\PY{p}{,} \PY{n}{recreated\PYZus{}image}\PY{p}{)}
                         
                         
                         \PY{k}{return} \PY{n}{name\PYZus{}noise}\PY{p}{,} \PY{n}{name\PYZus{}image}
         
                 \PY{k}{return} \PY{l+m+mi}{1}
\end{Verbatim}


    \paragraph{Test your implementation by running the below cell and
visualizing. The first image is the original input image, second image
is the noise added, and third is the adversarially perturbed
image.}\label{test-your-implementation-by-running-the-below-cell-and-visualizing.-the-first-image-is-the-original-input-image-second-image-is-the-noise-added-and-third-is-the-adversarially-perturbed-image.}

    \begin{Verbatim}[commandchars=\\\{\}]
{\color{incolor}In [{\color{incolor}52}]:} \PY{n}{target\PYZus{}example} \PY{o}{=} \PY{l+m+mi}{2}
         \PY{p}{(}\PY{n}{img\PYZus{}path}\PY{p}{,} \PY{n}{original\PYZus{}image}\PY{p}{,} \PY{n}{prep\PYZus{}img}\PY{p}{,} \PY{n}{target\PYZus{}class}\PY{p}{,} \PY{n}{\PYZus{}}\PY{p}{,} \PY{n}{pretrained\PYZus{}model}\PY{p}{)} \PY{o}{=}\PYZbs{}
             \PY{n}{get\PYZus{}params}\PY{p}{(}\PY{n}{target\PYZus{}example}\PY{p}{)}
         
         \PY{n}{FGS\PYZus{}untargeted} \PY{o}{=} \PY{n}{FastGradientSignUntargeted}\PY{p}{(}\PY{n}{pretrained\PYZus{}model}\PY{p}{,} \PY{l+m+mf}{0.01}\PY{p}{)}
         \PY{n}{name\PYZus{}noise}\PY{p}{,} \PY{n}{name\PYZus{}image} \PY{o}{=} \PY{n}{FGS\PYZus{}untargeted}\PY{o}{.}\PY{n}{generate}\PY{p}{(}\PY{n}{original\PYZus{}image}\PY{p}{,} \PY{n}{target\PYZus{}class}\PY{p}{)}
         
         \PY{n}{original} \PY{o}{=} \PY{n}{Image}\PY{p}{(}\PY{n}{img\PYZus{}path}\PY{p}{)}
         \PY{n}{noise} \PY{o}{=} \PY{n}{Image}\PY{p}{(}\PY{n}{name\PYZus{}noise}\PY{p}{)}
         \PY{n}{adversarial} \PY{o}{=} \PY{n}{Image}\PY{p}{(}\PY{n}{name\PYZus{}image}\PY{p}{)}
         
         \PY{n}{display}\PY{p}{(}\PY{n}{original}\PY{p}{,} \PY{n}{noise}\PY{p}{,} \PY{n}{adversarial}\PY{p}{)}
\end{Verbatim}


    \begin{Verbatim}[commandchars=\\\{\}]
Iteration: 0
Original image was predicted as: 13 with adversarial noise converted to: 19 and predicted with confidence of: 0.96890265

    \end{Verbatim}

    \begin{Verbatim}[commandchars=\\\{\}]
/Users/gautam/anaconda3/lib/python3.6/site-packages/ipykernel\_launcher.py:58: UserWarning: Implicit dimension choice for softmax has been deprecated. Change the call to include dim=X as an argument.

    \end{Verbatim}

    \begin{center}
    \adjustimage{max size={0.9\linewidth}{0.9\paperheight}}{output_9_2.jpeg}
    \end{center}
    { \hspace*{\fill} \\}
    
    \begin{center}
    \adjustimage{max size={0.9\linewidth}{0.9\paperheight}}{output_9_3.jpeg}
    \end{center}
    { \hspace*{\fill} \\}
    
    \begin{center}
    \adjustimage{max size={0.9\linewidth}{0.9\paperheight}}{output_9_4.jpeg}
    \end{center}
    { \hspace*{\fill} \\}
    
    What you did above was an untargeted FGSM, so what it did was make a
small change to fool the classifier into thinking the image was labeled
with any other label than the one it actually is. On the other hand,
\textsubscript{targeted} FGSM is where the adversarial image that is
created is of a very deliberate, specific label. Besides that, the same
techniques are applied. Read through the code below and fill in the 5
blank spots with very similar code to what you filled in above.

    \begin{Verbatim}[commandchars=\\\{\}]
{\color{incolor}In [{\color{incolor}53}]:} \PY{k}{class} \PY{n+nc}{FastGradientSignTargeted}\PY{p}{(}\PY{p}{)}\PY{p}{:}
             \PY{l+s+sd}{\PYZdq{}\PYZdq{}\PYZdq{}}
         \PY{l+s+sd}{        Fast gradient sign untargeted adversarial attack, maximizes the target class activation}
         \PY{l+s+sd}{        with iterative grad sign updates}
         \PY{l+s+sd}{    \PYZdq{}\PYZdq{}\PYZdq{}}
             \PY{k}{def} \PY{n+nf}{\PYZus{}\PYZus{}init\PYZus{}\PYZus{}}\PY{p}{(}\PY{n+nb+bp}{self}\PY{p}{,} \PY{n}{model}\PY{p}{,} \PY{n}{alpha}\PY{p}{)}\PY{p}{:}
                 \PY{n+nb+bp}{self}\PY{o}{.}\PY{n}{model} \PY{o}{=} \PY{n}{model}
                 \PY{n+nb+bp}{self}\PY{o}{.}\PY{n}{model}\PY{o}{.}\PY{n}{eval}\PY{p}{(}\PY{p}{)}
                 \PY{c+c1}{\PYZsh{} Movement multiplier per iteration}
                 \PY{n+nb+bp}{self}\PY{o}{.}\PY{n}{alpha} \PY{o}{=} \PY{n}{alpha}
                 \PY{c+c1}{\PYZsh{} Create the folder to export images if not exists}
                 \PY{k}{if} \PY{o+ow}{not} \PY{n}{os}\PY{o}{.}\PY{n}{path}\PY{o}{.}\PY{n}{exists}\PY{p}{(}\PY{l+s+s1}{\PYZsq{}}\PY{l+s+s1}{./generated}\PY{l+s+s1}{\PYZsq{}}\PY{p}{)}\PY{p}{:}
                     \PY{n}{os}\PY{o}{.}\PY{n}{makedirs}\PY{p}{(}\PY{l+s+s1}{\PYZsq{}}\PY{l+s+s1}{./generated}\PY{l+s+s1}{\PYZsq{}}\PY{p}{)}
         
             \PY{k}{def} \PY{n+nf}{generate}\PY{p}{(}\PY{n+nb+bp}{self}\PY{p}{,} \PY{n}{original\PYZus{}image}\PY{p}{,} \PY{n}{org\PYZus{}class}\PY{p}{,} \PY{n}{target\PYZus{}class}\PY{p}{)}\PY{p}{:}
                 \PY{c+c1}{\PYZsh{} I honestly dont know a better way to create a variable with specific value}
                 \PY{c+c1}{\PYZsh{} Targeting the specific class}
                 \PY{n}{im\PYZus{}label\PYZus{}as\PYZus{}var} \PY{o}{=} \PY{n}{Variable}\PY{p}{(}\PY{n}{torch}\PY{o}{.}\PY{n}{from\PYZus{}numpy}\PY{p}{(}\PY{n}{np}\PY{o}{.}\PY{n}{asarray}\PY{p}{(}\PY{p}{[}\PY{n}{target\PYZus{}class}\PY{p}{]}\PY{p}{)}\PY{p}{)}\PY{p}{)}
                 \PY{c+c1}{\PYZsh{} Define loss functions}
                 \PY{n}{ce\PYZus{}loss} \PY{o}{=} \PY{n}{nn}\PY{o}{.}\PY{n}{CrossEntropyLoss}\PY{p}{(}\PY{p}{)}
                 \PY{c+c1}{\PYZsh{} Process image}
                 \PY{n}{processed\PYZus{}image} \PY{o}{=} \PY{n}{preprocess\PYZus{}image}\PY{p}{(}\PY{n}{original\PYZus{}image}\PY{p}{)}
                 \PY{c+c1}{\PYZsh{} Start iteration}
                 \PY{k}{for} \PY{n}{i} \PY{o+ow}{in} \PY{n+nb}{range}\PY{p}{(}\PY{l+m+mi}{100}\PY{p}{)}\PY{p}{:}
                     \PY{n+nb}{print}\PY{p}{(}\PY{l+s+s1}{\PYZsq{}}\PY{l+s+s1}{Iteration:}\PY{l+s+s1}{\PYZsq{}}\PY{p}{,} \PY{n+nb}{str}\PY{p}{(}\PY{n}{i}\PY{p}{)}\PY{p}{)}
                     \PY{c+c1}{\PYZsh{} zero\PYZus{}gradients(x)}
                     \PY{c+c1}{\PYZsh{} Zero out previous gradients}
                     \PY{c+c1}{\PYZsh{} Can also use zero\PYZus{}gradients(x)}
                     \PY{n}{processed\PYZus{}image}\PY{o}{.}\PY{n}{grad} \PY{o}{=} \PY{k+kc}{None}
                     \PY{c+c1}{\PYZsh{} Forward pass}
                     \PY{n}{model\PYZus{}output} \PY{o}{=} \PY{n+nb+bp}{self}\PY{o}{.}\PY{n}{model}\PY{p}{(}\PY{n}{processed\PYZus{}image}\PY{p}{)}
                     \PY{c+c1}{\PYZsh{} Calculate CE loss}
                     \PY{n}{pred\PYZus{}loss} \PY{o}{=} \PY{n}{ce\PYZus{}loss}\PY{p}{(}\PY{n}{model\PYZus{}output}\PY{p}{,} \PY{n}{im\PYZus{}label\PYZus{}as\PYZus{}var}\PY{p}{)}
                     \PY{c+c1}{\PYZsh{} Do backward pass}
                     \PY{n}{pred\PYZus{}loss}\PY{o}{.}\PY{n}{backward}\PY{p}{(}\PY{p}{)}
                     \PY{c+c1}{\PYZsh{} Create Noise}
                     \PY{c+c1}{\PYZsh{} Here, processed\PYZus{}image.grad.data is also the same thing is the backward gradient from}
                     \PY{c+c1}{\PYZsh{} the first layer, can use that with hooks as well}
                     \PY{n}{adv\PYZus{}noise} \PY{o}{=} \PY{n+nb+bp}{self}\PY{o}{.}\PY{n}{alpha} \PY{o}{*} \PY{n}{torch}\PY{o}{.}\PY{n}{sign}\PY{p}{(}\PY{n}{processed\PYZus{}image}\PY{o}{.}\PY{n}{grad}\PY{o}{.}\PY{n}{data}\PY{p}{)}
                     \PY{c+c1}{\PYZsh{} Subtract noise to processed image}
                     \PY{n}{processed\PYZus{}image}\PY{o}{.}\PY{n}{data} \PY{o}{=} \PY{n}{processed\PYZus{}image} \PY{o}{\PYZhy{}} \PY{n}{adv\PYZus{}noise}
         
                     \PY{c+c1}{\PYZsh{} Confirming if the image is indeed adversarial with added noise}
                     \PY{c+c1}{\PYZsh{} This is necessary (for some cases) because when we recreate image}
                     \PY{c+c1}{\PYZsh{} the values become integers between 1 and 255 and sometimes the adversariality}
                     \PY{c+c1}{\PYZsh{} is lost in the recreation process}
         
                     \PY{c+c1}{\PYZsh{} Generate confirmation image}
                     \PY{n}{recreated\PYZus{}image} \PY{o}{=} \PY{n}{recreate\PYZus{}image}\PY{p}{(}\PY{n}{processed\PYZus{}image}\PY{p}{)}
                     \PY{c+c1}{\PYZsh{} Process confirmation image}
                     \PY{n}{prep\PYZus{}confirmation\PYZus{}image} \PY{o}{=} \PY{n}{preprocess\PYZus{}image}\PY{p}{(}\PY{n}{recreated\PYZus{}image}\PY{p}{)}
                     \PY{c+c1}{\PYZsh{} Forward pass}
                     \PY{n}{confirmation\PYZus{}out} \PY{o}{=} \PY{n+nb+bp}{self}\PY{o}{.}\PY{n}{model}\PY{p}{(}\PY{n}{prep\PYZus{}confirmation\PYZus{}image}\PY{p}{)}
                     \PY{c+c1}{\PYZsh{} Get prediction}
                     \PY{n}{\PYZus{}}\PY{p}{,} \PY{n}{confirmation\PYZus{}prediction} \PY{o}{=} \PY{n}{confirmation\PYZus{}out}\PY{o}{.}\PY{n}{data}\PY{o}{.}\PY{n}{max}\PY{p}{(}\PY{l+m+mi}{1}\PY{p}{)}
                     \PY{c+c1}{\PYZsh{} Get Probability}
                     \PY{n}{confirmation\PYZus{}confidence} \PY{o}{=} \PYZbs{}
                         \PY{n}{nn}\PY{o}{.}\PY{n}{functional}\PY{o}{.}\PY{n}{softmax}\PY{p}{(}\PY{n}{confirmation\PYZus{}out}\PY{p}{)}\PY{p}{[}\PY{l+m+mi}{0}\PY{p}{]}\PY{p}{[}\PY{n}{confirmation\PYZus{}prediction}\PY{p}{]}\PY{o}{.}\PY{n}{data}\PY{o}{.}\PY{n}{numpy}\PY{p}{(}\PY{p}{)}\PY{p}{[}\PY{l+m+mi}{0}\PY{p}{]}
                     \PY{c+c1}{\PYZsh{} Convert tensor to int}
                     \PY{n}{confirmation\PYZus{}prediction} \PY{o}{=} \PY{n}{confirmation\PYZus{}prediction}\PY{o}{.}\PY{n}{numpy}\PY{p}{(}\PY{p}{)}\PY{p}{[}\PY{l+m+mi}{0}\PY{p}{]}
                     \PY{c+c1}{\PYZsh{} Check if the prediction is different than the original}
                     \PY{k}{if} \PY{n}{confirmation\PYZus{}prediction} \PY{o}{==} \PY{n}{target\PYZus{}class}\PY{p}{:}
                         \PY{n+nb}{print}\PY{p}{(}\PY{l+s+s1}{\PYZsq{}}\PY{l+s+s1}{Original image was predicted as:}\PY{l+s+s1}{\PYZsq{}}\PY{p}{,} \PY{n}{org\PYZus{}class}\PY{p}{,}
                               \PY{l+s+s1}{\PYZsq{}}\PY{l+s+s1}{with adversarial noise converted to:}\PY{l+s+s1}{\PYZsq{}}\PY{p}{,} \PY{n}{confirmation\PYZus{}prediction}\PY{p}{,}
                               \PY{l+s+s1}{\PYZsq{}}\PY{l+s+s1}{and predicted with confidence of:}\PY{l+s+s1}{\PYZsq{}}\PY{p}{,} \PY{n}{confirmation\PYZus{}confidence}\PY{p}{)}
                         \PY{c+c1}{\PYZsh{} Create the image for noise as: Original image \PYZhy{} generated image}
                         \PY{n}{noise\PYZus{}image} \PY{o}{=} \PY{n}{confirmation\PYZus{}prediction} \PY{o}{\PYZhy{}} \PY{n}{original\PYZus{}image}
                         \PY{n}{name\PYZus{}noise} \PY{o}{=} \PY{l+s+s1}{\PYZsq{}}\PY{l+s+s1}{./generated/targeted\PYZus{}adv\PYZus{}noise\PYZus{}from\PYZus{}}\PY{l+s+s1}{\PYZsq{}} \PY{o}{+} \PY{n+nb}{str}\PY{p}{(}\PY{n}{org\PYZus{}class}\PY{p}{)} \PY{o}{+} \PY{l+s+s1}{\PYZsq{}}\PY{l+s+s1}{\PYZus{}to\PYZus{}}\PY{l+s+s1}{\PYZsq{}} \PY{o}{+} \PY{n+nb}{str}\PY{p}{(}\PY{n}{confirmation\PYZus{}prediction}\PY{p}{)} \PY{o}{+} \PY{l+s+s1}{\PYZsq{}}\PY{l+s+s1}{.jpg}\PY{l+s+s1}{\PYZsq{}}
                         \PY{n}{cv2}\PY{o}{.}\PY{n}{imwrite}\PY{p}{(}\PY{n}{name\PYZus{}noise}\PY{p}{,} \PY{n}{noise\PYZus{}image}\PY{p}{)}
                         \PY{c+c1}{\PYZsh{} Write image}
                         \PY{n}{name\PYZus{}image} \PY{o}{=} \PY{l+s+s1}{\PYZsq{}}\PY{l+s+s1}{./generated/targeted\PYZus{}adv\PYZus{}img\PYZus{}from\PYZus{}}\PY{l+s+s1}{\PYZsq{}} \PY{o}{+} \PY{n+nb}{str}\PY{p}{(}\PY{n}{org\PYZus{}class}\PY{p}{)} \PY{o}{+} \PY{l+s+s1}{\PYZsq{}}\PY{l+s+s1}{\PYZus{}to\PYZus{}}\PY{l+s+s1}{\PYZsq{}} \PY{o}{+} \PY{n+nb}{str}\PY{p}{(}\PY{n}{confirmation\PYZus{}prediction}\PY{p}{)} \PY{o}{+} \PY{l+s+s1}{\PYZsq{}}\PY{l+s+s1}{.jpg}\PY{l+s+s1}{\PYZsq{}}
                         \PY{n}{cv2}\PY{o}{.}\PY{n}{imwrite}\PY{p}{(}\PY{n}{name\PYZus{}image}\PY{p}{,} \PY{n}{recreated\PYZus{}image}\PY{p}{)}
                         \PY{k}{return} \PY{n}{name\PYZus{}noise}\PY{p}{,} \PY{n}{name\PYZus{}image}
                         \PY{k}{break}
         
                 \PY{k}{return} \PY{l+m+mi}{1}
\end{Verbatim}


    \begin{Verbatim}[commandchars=\\\{\}]
{\color{incolor}In [{\color{incolor}54}]:} \PY{n}{target\PYZus{}example} \PY{o}{=} \PY{l+m+mi}{1}  \PY{c+c1}{\PYZsh{} Apple}
         \PY{p}{(}\PY{n}{img\PYZus{}path}\PY{p}{,} \PY{n}{original\PYZus{}image}\PY{p}{,} \PY{n}{prep\PYZus{}img}\PY{p}{,} \PY{n}{org\PYZus{}class}\PY{p}{,} \PY{n}{\PYZus{}}\PY{p}{,} \PY{n}{pretrained\PYZus{}model}\PY{p}{)} \PY{o}{=}\PYZbs{}
             \PY{n}{get\PYZus{}params}\PY{p}{(}\PY{n}{target\PYZus{}example}\PY{p}{)}
         \PY{n}{target\PYZus{}class} \PY{o}{=} \PY{l+m+mi}{62}  \PY{c+c1}{\PYZsh{} Mud turtle}
         
         \PY{n}{FGS\PYZus{}untargeted} \PY{o}{=} \PY{n}{FastGradientSignTargeted}\PY{p}{(}\PY{n}{pretrained\PYZus{}model}\PY{p}{,} \PY{l+m+mf}{0.01}\PY{p}{)}
         \PY{n}{name\PYZus{}noise}\PY{p}{,} \PY{n}{name\PYZus{}image} \PY{o}{=} \PY{n}{FGS\PYZus{}untargeted}\PY{o}{.}\PY{n}{generate}\PY{p}{(}\PY{n}{original\PYZus{}image}\PY{p}{,} \PY{n}{org\PYZus{}class}\PY{p}{,} \PY{n}{target\PYZus{}class}\PY{p}{)}
         
         \PY{n}{original} \PY{o}{=} \PY{n}{Image}\PY{p}{(}\PY{n}{img\PYZus{}path}\PY{p}{)}
         \PY{n}{noise} \PY{o}{=} \PY{n}{Image}\PY{p}{(}\PY{n}{name\PYZus{}noise}\PY{p}{)}
         \PY{n}{adversarial} \PY{o}{=} \PY{n}{Image}\PY{p}{(}\PY{n}{name\PYZus{}image}\PY{p}{)}
         
         \PY{n}{display}\PY{p}{(}\PY{n}{original}\PY{p}{,} \PY{n}{noise}\PY{p}{,} \PY{n}{adversarial}\PY{p}{)}
\end{Verbatim}


    \begin{Verbatim}[commandchars=\\\{\}]
Iteration: 0

    \end{Verbatim}

    \begin{Verbatim}[commandchars=\\\{\}]
/Users/gautam/anaconda3/lib/python3.6/site-packages/ipykernel\_launcher.py:57: UserWarning: Implicit dimension choice for softmax has been deprecated. Change the call to include dim=X as an argument.

    \end{Verbatim}

    \begin{Verbatim}[commandchars=\\\{\}]
Iteration: 1
Iteration: 2
Iteration: 3
Iteration: 4
Original image was predicted as: 390 with adversarial noise converted to: 62 and predicted with confidence of: 0.27861667

    \end{Verbatim}

    \begin{center}
    \adjustimage{max size={0.9\linewidth}{0.9\paperheight}}{output_12_3.jpeg}
    \end{center}
    { \hspace*{\fill} \\}
    
    \begin{center}
    \adjustimage{max size={0.9\linewidth}{0.9\paperheight}}{output_12_4.jpeg}
    \end{center}
    { \hspace*{\fill} \\}
    
    \begin{center}
    \adjustimage{max size={0.9\linewidth}{0.9\paperheight}}{output_12_5.jpeg}
    \end{center}
    { \hspace*{\fill} \\}
    

    % Add a bibliography block to the postdoc
    
    
    
    \end{document}
