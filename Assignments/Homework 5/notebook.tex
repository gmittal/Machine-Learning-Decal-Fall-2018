
% Default to the notebook output style

    


% Inherit from the specified cell style.




    
\documentclass[11pt]{article}

    
    
    \usepackage[T1]{fontenc}
    % Nicer default font (+ math font) than Computer Modern for most use cases
    \usepackage{mathpazo}

    % Basic figure setup, for now with no caption control since it's done
    % automatically by Pandoc (which extracts ![](path) syntax from Markdown).
    \usepackage{graphicx}
    % We will generate all images so they have a width \maxwidth. This means
    % that they will get their normal width if they fit onto the page, but
    % are scaled down if they would overflow the margins.
    \makeatletter
    \def\maxwidth{\ifdim\Gin@nat@width>\linewidth\linewidth
    \else\Gin@nat@width\fi}
    \makeatother
    \let\Oldincludegraphics\includegraphics
    % Set max figure width to be 80% of text width, for now hardcoded.
    \renewcommand{\includegraphics}[1]{\Oldincludegraphics[width=.8\maxwidth]{#1}}
    % Ensure that by default, figures have no caption (until we provide a
    % proper Figure object with a Caption API and a way to capture that
    % in the conversion process - todo).
    \usepackage{caption}
    \DeclareCaptionLabelFormat{nolabel}{}
    \captionsetup{labelformat=nolabel}

    \usepackage{adjustbox} % Used to constrain images to a maximum size 
    \usepackage{xcolor} % Allow colors to be defined
    \usepackage{enumerate} % Needed for markdown enumerations to work
    \usepackage{geometry} % Used to adjust the document margins
    \usepackage{amsmath} % Equations
    \usepackage{amssymb} % Equations
    \usepackage{textcomp} % defines textquotesingle
    % Hack from http://tex.stackexchange.com/a/47451/13684:
    \AtBeginDocument{%
        \def\PYZsq{\textquotesingle}% Upright quotes in Pygmentized code
    }
    \usepackage{upquote} % Upright quotes for verbatim code
    \usepackage{eurosym} % defines \euro
    \usepackage[mathletters]{ucs} % Extended unicode (utf-8) support
    \usepackage[utf8x]{inputenc} % Allow utf-8 characters in the tex document
    \usepackage{fancyvrb} % verbatim replacement that allows latex
    \usepackage{grffile} % extends the file name processing of package graphics 
                         % to support a larger range 
    % The hyperref package gives us a pdf with properly built
    % internal navigation ('pdf bookmarks' for the table of contents,
    % internal cross-reference links, web links for URLs, etc.)
    \usepackage{hyperref}
    \usepackage{longtable} % longtable support required by pandoc >1.10
    \usepackage{booktabs}  % table support for pandoc > 1.12.2
    \usepackage[inline]{enumitem} % IRkernel/repr support (it uses the enumerate* environment)
    \usepackage[normalem]{ulem} % ulem is needed to support strikethroughs (\sout)
                                % normalem makes italics be italics, not underlines
    

    
    
    % Colors for the hyperref package
    \definecolor{urlcolor}{rgb}{0,.145,.698}
    \definecolor{linkcolor}{rgb}{.71,0.21,0.01}
    \definecolor{citecolor}{rgb}{.12,.54,.11}

    % ANSI colors
    \definecolor{ansi-black}{HTML}{3E424D}
    \definecolor{ansi-black-intense}{HTML}{282C36}
    \definecolor{ansi-red}{HTML}{E75C58}
    \definecolor{ansi-red-intense}{HTML}{B22B31}
    \definecolor{ansi-green}{HTML}{00A250}
    \definecolor{ansi-green-intense}{HTML}{007427}
    \definecolor{ansi-yellow}{HTML}{DDB62B}
    \definecolor{ansi-yellow-intense}{HTML}{B27D12}
    \definecolor{ansi-blue}{HTML}{208FFB}
    \definecolor{ansi-blue-intense}{HTML}{0065CA}
    \definecolor{ansi-magenta}{HTML}{D160C4}
    \definecolor{ansi-magenta-intense}{HTML}{A03196}
    \definecolor{ansi-cyan}{HTML}{60C6C8}
    \definecolor{ansi-cyan-intense}{HTML}{258F8F}
    \definecolor{ansi-white}{HTML}{C5C1B4}
    \definecolor{ansi-white-intense}{HTML}{A1A6B2}

    % commands and environments needed by pandoc snippets
    % extracted from the output of `pandoc -s`
    \providecommand{\tightlist}{%
      \setlength{\itemsep}{0pt}\setlength{\parskip}{0pt}}
    \DefineVerbatimEnvironment{Highlighting}{Verbatim}{commandchars=\\\{\}}
    % Add ',fontsize=\small' for more characters per line
    \newenvironment{Shaded}{}{}
    \newcommand{\KeywordTok}[1]{\textcolor[rgb]{0.00,0.44,0.13}{\textbf{{#1}}}}
    \newcommand{\DataTypeTok}[1]{\textcolor[rgb]{0.56,0.13,0.00}{{#1}}}
    \newcommand{\DecValTok}[1]{\textcolor[rgb]{0.25,0.63,0.44}{{#1}}}
    \newcommand{\BaseNTok}[1]{\textcolor[rgb]{0.25,0.63,0.44}{{#1}}}
    \newcommand{\FloatTok}[1]{\textcolor[rgb]{0.25,0.63,0.44}{{#1}}}
    \newcommand{\CharTok}[1]{\textcolor[rgb]{0.25,0.44,0.63}{{#1}}}
    \newcommand{\StringTok}[1]{\textcolor[rgb]{0.25,0.44,0.63}{{#1}}}
    \newcommand{\CommentTok}[1]{\textcolor[rgb]{0.38,0.63,0.69}{\textit{{#1}}}}
    \newcommand{\OtherTok}[1]{\textcolor[rgb]{0.00,0.44,0.13}{{#1}}}
    \newcommand{\AlertTok}[1]{\textcolor[rgb]{1.00,0.00,0.00}{\textbf{{#1}}}}
    \newcommand{\FunctionTok}[1]{\textcolor[rgb]{0.02,0.16,0.49}{{#1}}}
    \newcommand{\RegionMarkerTok}[1]{{#1}}
    \newcommand{\ErrorTok}[1]{\textcolor[rgb]{1.00,0.00,0.00}{\textbf{{#1}}}}
    \newcommand{\NormalTok}[1]{{#1}}
    
    % Additional commands for more recent versions of Pandoc
    \newcommand{\ConstantTok}[1]{\textcolor[rgb]{0.53,0.00,0.00}{{#1}}}
    \newcommand{\SpecialCharTok}[1]{\textcolor[rgb]{0.25,0.44,0.63}{{#1}}}
    \newcommand{\VerbatimStringTok}[1]{\textcolor[rgb]{0.25,0.44,0.63}{{#1}}}
    \newcommand{\SpecialStringTok}[1]{\textcolor[rgb]{0.73,0.40,0.53}{{#1}}}
    \newcommand{\ImportTok}[1]{{#1}}
    \newcommand{\DocumentationTok}[1]{\textcolor[rgb]{0.73,0.13,0.13}{\textit{{#1}}}}
    \newcommand{\AnnotationTok}[1]{\textcolor[rgb]{0.38,0.63,0.69}{\textbf{\textit{{#1}}}}}
    \newcommand{\CommentVarTok}[1]{\textcolor[rgb]{0.38,0.63,0.69}{\textbf{\textit{{#1}}}}}
    \newcommand{\VariableTok}[1]{\textcolor[rgb]{0.10,0.09,0.49}{{#1}}}
    \newcommand{\ControlFlowTok}[1]{\textcolor[rgb]{0.00,0.44,0.13}{\textbf{{#1}}}}
    \newcommand{\OperatorTok}[1]{\textcolor[rgb]{0.40,0.40,0.40}{{#1}}}
    \newcommand{\BuiltInTok}[1]{{#1}}
    \newcommand{\ExtensionTok}[1]{{#1}}
    \newcommand{\PreprocessorTok}[1]{\textcolor[rgb]{0.74,0.48,0.00}{{#1}}}
    \newcommand{\AttributeTok}[1]{\textcolor[rgb]{0.49,0.56,0.16}{{#1}}}
    \newcommand{\InformationTok}[1]{\textcolor[rgb]{0.38,0.63,0.69}{\textbf{\textit{{#1}}}}}
    \newcommand{\WarningTok}[1]{\textcolor[rgb]{0.38,0.63,0.69}{\textbf{\textit{{#1}}}}}
    
    
    % Define a nice break command that doesn't care if a line doesn't already
    % exist.
    \def\br{\hspace*{\fill} \\* }
    % Math Jax compatability definitions
    \def\gt{>}
    \def\lt{<}
    % Document parameters
    \title{DeepDream}
    
    
    

    % Pygments definitions
    
\makeatletter
\def\PY@reset{\let\PY@it=\relax \let\PY@bf=\relax%
    \let\PY@ul=\relax \let\PY@tc=\relax%
    \let\PY@bc=\relax \let\PY@ff=\relax}
\def\PY@tok#1{\csname PY@tok@#1\endcsname}
\def\PY@toks#1+{\ifx\relax#1\empty\else%
    \PY@tok{#1}\expandafter\PY@toks\fi}
\def\PY@do#1{\PY@bc{\PY@tc{\PY@ul{%
    \PY@it{\PY@bf{\PY@ff{#1}}}}}}}
\def\PY#1#2{\PY@reset\PY@toks#1+\relax+\PY@do{#2}}

\expandafter\def\csname PY@tok@w\endcsname{\def\PY@tc##1{\textcolor[rgb]{0.73,0.73,0.73}{##1}}}
\expandafter\def\csname PY@tok@c\endcsname{\let\PY@it=\textit\def\PY@tc##1{\textcolor[rgb]{0.25,0.50,0.50}{##1}}}
\expandafter\def\csname PY@tok@cp\endcsname{\def\PY@tc##1{\textcolor[rgb]{0.74,0.48,0.00}{##1}}}
\expandafter\def\csname PY@tok@k\endcsname{\let\PY@bf=\textbf\def\PY@tc##1{\textcolor[rgb]{0.00,0.50,0.00}{##1}}}
\expandafter\def\csname PY@tok@kp\endcsname{\def\PY@tc##1{\textcolor[rgb]{0.00,0.50,0.00}{##1}}}
\expandafter\def\csname PY@tok@kt\endcsname{\def\PY@tc##1{\textcolor[rgb]{0.69,0.00,0.25}{##1}}}
\expandafter\def\csname PY@tok@o\endcsname{\def\PY@tc##1{\textcolor[rgb]{0.40,0.40,0.40}{##1}}}
\expandafter\def\csname PY@tok@ow\endcsname{\let\PY@bf=\textbf\def\PY@tc##1{\textcolor[rgb]{0.67,0.13,1.00}{##1}}}
\expandafter\def\csname PY@tok@nb\endcsname{\def\PY@tc##1{\textcolor[rgb]{0.00,0.50,0.00}{##1}}}
\expandafter\def\csname PY@tok@nf\endcsname{\def\PY@tc##1{\textcolor[rgb]{0.00,0.00,1.00}{##1}}}
\expandafter\def\csname PY@tok@nc\endcsname{\let\PY@bf=\textbf\def\PY@tc##1{\textcolor[rgb]{0.00,0.00,1.00}{##1}}}
\expandafter\def\csname PY@tok@nn\endcsname{\let\PY@bf=\textbf\def\PY@tc##1{\textcolor[rgb]{0.00,0.00,1.00}{##1}}}
\expandafter\def\csname PY@tok@ne\endcsname{\let\PY@bf=\textbf\def\PY@tc##1{\textcolor[rgb]{0.82,0.25,0.23}{##1}}}
\expandafter\def\csname PY@tok@nv\endcsname{\def\PY@tc##1{\textcolor[rgb]{0.10,0.09,0.49}{##1}}}
\expandafter\def\csname PY@tok@no\endcsname{\def\PY@tc##1{\textcolor[rgb]{0.53,0.00,0.00}{##1}}}
\expandafter\def\csname PY@tok@nl\endcsname{\def\PY@tc##1{\textcolor[rgb]{0.63,0.63,0.00}{##1}}}
\expandafter\def\csname PY@tok@ni\endcsname{\let\PY@bf=\textbf\def\PY@tc##1{\textcolor[rgb]{0.60,0.60,0.60}{##1}}}
\expandafter\def\csname PY@tok@na\endcsname{\def\PY@tc##1{\textcolor[rgb]{0.49,0.56,0.16}{##1}}}
\expandafter\def\csname PY@tok@nt\endcsname{\let\PY@bf=\textbf\def\PY@tc##1{\textcolor[rgb]{0.00,0.50,0.00}{##1}}}
\expandafter\def\csname PY@tok@nd\endcsname{\def\PY@tc##1{\textcolor[rgb]{0.67,0.13,1.00}{##1}}}
\expandafter\def\csname PY@tok@s\endcsname{\def\PY@tc##1{\textcolor[rgb]{0.73,0.13,0.13}{##1}}}
\expandafter\def\csname PY@tok@sd\endcsname{\let\PY@it=\textit\def\PY@tc##1{\textcolor[rgb]{0.73,0.13,0.13}{##1}}}
\expandafter\def\csname PY@tok@si\endcsname{\let\PY@bf=\textbf\def\PY@tc##1{\textcolor[rgb]{0.73,0.40,0.53}{##1}}}
\expandafter\def\csname PY@tok@se\endcsname{\let\PY@bf=\textbf\def\PY@tc##1{\textcolor[rgb]{0.73,0.40,0.13}{##1}}}
\expandafter\def\csname PY@tok@sr\endcsname{\def\PY@tc##1{\textcolor[rgb]{0.73,0.40,0.53}{##1}}}
\expandafter\def\csname PY@tok@ss\endcsname{\def\PY@tc##1{\textcolor[rgb]{0.10,0.09,0.49}{##1}}}
\expandafter\def\csname PY@tok@sx\endcsname{\def\PY@tc##1{\textcolor[rgb]{0.00,0.50,0.00}{##1}}}
\expandafter\def\csname PY@tok@m\endcsname{\def\PY@tc##1{\textcolor[rgb]{0.40,0.40,0.40}{##1}}}
\expandafter\def\csname PY@tok@gh\endcsname{\let\PY@bf=\textbf\def\PY@tc##1{\textcolor[rgb]{0.00,0.00,0.50}{##1}}}
\expandafter\def\csname PY@tok@gu\endcsname{\let\PY@bf=\textbf\def\PY@tc##1{\textcolor[rgb]{0.50,0.00,0.50}{##1}}}
\expandafter\def\csname PY@tok@gd\endcsname{\def\PY@tc##1{\textcolor[rgb]{0.63,0.00,0.00}{##1}}}
\expandafter\def\csname PY@tok@gi\endcsname{\def\PY@tc##1{\textcolor[rgb]{0.00,0.63,0.00}{##1}}}
\expandafter\def\csname PY@tok@gr\endcsname{\def\PY@tc##1{\textcolor[rgb]{1.00,0.00,0.00}{##1}}}
\expandafter\def\csname PY@tok@ge\endcsname{\let\PY@it=\textit}
\expandafter\def\csname PY@tok@gs\endcsname{\let\PY@bf=\textbf}
\expandafter\def\csname PY@tok@gp\endcsname{\let\PY@bf=\textbf\def\PY@tc##1{\textcolor[rgb]{0.00,0.00,0.50}{##1}}}
\expandafter\def\csname PY@tok@go\endcsname{\def\PY@tc##1{\textcolor[rgb]{0.53,0.53,0.53}{##1}}}
\expandafter\def\csname PY@tok@gt\endcsname{\def\PY@tc##1{\textcolor[rgb]{0.00,0.27,0.87}{##1}}}
\expandafter\def\csname PY@tok@err\endcsname{\def\PY@bc##1{\setlength{\fboxsep}{0pt}\fcolorbox[rgb]{1.00,0.00,0.00}{1,1,1}{\strut ##1}}}
\expandafter\def\csname PY@tok@kc\endcsname{\let\PY@bf=\textbf\def\PY@tc##1{\textcolor[rgb]{0.00,0.50,0.00}{##1}}}
\expandafter\def\csname PY@tok@kd\endcsname{\let\PY@bf=\textbf\def\PY@tc##1{\textcolor[rgb]{0.00,0.50,0.00}{##1}}}
\expandafter\def\csname PY@tok@kn\endcsname{\let\PY@bf=\textbf\def\PY@tc##1{\textcolor[rgb]{0.00,0.50,0.00}{##1}}}
\expandafter\def\csname PY@tok@kr\endcsname{\let\PY@bf=\textbf\def\PY@tc##1{\textcolor[rgb]{0.00,0.50,0.00}{##1}}}
\expandafter\def\csname PY@tok@bp\endcsname{\def\PY@tc##1{\textcolor[rgb]{0.00,0.50,0.00}{##1}}}
\expandafter\def\csname PY@tok@fm\endcsname{\def\PY@tc##1{\textcolor[rgb]{0.00,0.00,1.00}{##1}}}
\expandafter\def\csname PY@tok@vc\endcsname{\def\PY@tc##1{\textcolor[rgb]{0.10,0.09,0.49}{##1}}}
\expandafter\def\csname PY@tok@vg\endcsname{\def\PY@tc##1{\textcolor[rgb]{0.10,0.09,0.49}{##1}}}
\expandafter\def\csname PY@tok@vi\endcsname{\def\PY@tc##1{\textcolor[rgb]{0.10,0.09,0.49}{##1}}}
\expandafter\def\csname PY@tok@vm\endcsname{\def\PY@tc##1{\textcolor[rgb]{0.10,0.09,0.49}{##1}}}
\expandafter\def\csname PY@tok@sa\endcsname{\def\PY@tc##1{\textcolor[rgb]{0.73,0.13,0.13}{##1}}}
\expandafter\def\csname PY@tok@sb\endcsname{\def\PY@tc##1{\textcolor[rgb]{0.73,0.13,0.13}{##1}}}
\expandafter\def\csname PY@tok@sc\endcsname{\def\PY@tc##1{\textcolor[rgb]{0.73,0.13,0.13}{##1}}}
\expandafter\def\csname PY@tok@dl\endcsname{\def\PY@tc##1{\textcolor[rgb]{0.73,0.13,0.13}{##1}}}
\expandafter\def\csname PY@tok@s2\endcsname{\def\PY@tc##1{\textcolor[rgb]{0.73,0.13,0.13}{##1}}}
\expandafter\def\csname PY@tok@sh\endcsname{\def\PY@tc##1{\textcolor[rgb]{0.73,0.13,0.13}{##1}}}
\expandafter\def\csname PY@tok@s1\endcsname{\def\PY@tc##1{\textcolor[rgb]{0.73,0.13,0.13}{##1}}}
\expandafter\def\csname PY@tok@mb\endcsname{\def\PY@tc##1{\textcolor[rgb]{0.40,0.40,0.40}{##1}}}
\expandafter\def\csname PY@tok@mf\endcsname{\def\PY@tc##1{\textcolor[rgb]{0.40,0.40,0.40}{##1}}}
\expandafter\def\csname PY@tok@mh\endcsname{\def\PY@tc##1{\textcolor[rgb]{0.40,0.40,0.40}{##1}}}
\expandafter\def\csname PY@tok@mi\endcsname{\def\PY@tc##1{\textcolor[rgb]{0.40,0.40,0.40}{##1}}}
\expandafter\def\csname PY@tok@il\endcsname{\def\PY@tc##1{\textcolor[rgb]{0.40,0.40,0.40}{##1}}}
\expandafter\def\csname PY@tok@mo\endcsname{\def\PY@tc##1{\textcolor[rgb]{0.40,0.40,0.40}{##1}}}
\expandafter\def\csname PY@tok@ch\endcsname{\let\PY@it=\textit\def\PY@tc##1{\textcolor[rgb]{0.25,0.50,0.50}{##1}}}
\expandafter\def\csname PY@tok@cm\endcsname{\let\PY@it=\textit\def\PY@tc##1{\textcolor[rgb]{0.25,0.50,0.50}{##1}}}
\expandafter\def\csname PY@tok@cpf\endcsname{\let\PY@it=\textit\def\PY@tc##1{\textcolor[rgb]{0.25,0.50,0.50}{##1}}}
\expandafter\def\csname PY@tok@c1\endcsname{\let\PY@it=\textit\def\PY@tc##1{\textcolor[rgb]{0.25,0.50,0.50}{##1}}}
\expandafter\def\csname PY@tok@cs\endcsname{\let\PY@it=\textit\def\PY@tc##1{\textcolor[rgb]{0.25,0.50,0.50}{##1}}}

\def\PYZbs{\char`\\}
\def\PYZus{\char`\_}
\def\PYZob{\char`\{}
\def\PYZcb{\char`\}}
\def\PYZca{\char`\^}
\def\PYZam{\char`\&}
\def\PYZlt{\char`\<}
\def\PYZgt{\char`\>}
\def\PYZsh{\char`\#}
\def\PYZpc{\char`\%}
\def\PYZdl{\char`\$}
\def\PYZhy{\char`\-}
\def\PYZsq{\char`\'}
\def\PYZdq{\char`\"}
\def\PYZti{\char`\~}
% for compatibility with earlier versions
\def\PYZat{@}
\def\PYZlb{[}
\def\PYZrb{]}
\makeatother


    % Exact colors from NB
    \definecolor{incolor}{rgb}{0.0, 0.0, 0.5}
    \definecolor{outcolor}{rgb}{0.545, 0.0, 0.0}



    
    % Prevent overflowing lines due to hard-to-break entities
    \sloppy 
    % Setup hyperref package
    \hypersetup{
      breaklinks=true,  % so long urls are correctly broken across lines
      colorlinks=true,
      urlcolor=urlcolor,
      linkcolor=linkcolor,
      citecolor=citecolor,
      }
    % Slightly bigger margins than the latex defaults
    
    \geometry{verbose,tmargin=1in,bmargin=1in,lmargin=1in,rmargin=1in}
    
    

    \begin{document}
    
    
    \maketitle
    
    

    
    \begin{Verbatim}[commandchars=\\\{\}]
{\color{incolor}In [{\color{incolor}1}]:} \PY{k+kn}{import} \PY{n+nn}{torch}
        \PY{k+kn}{import} \PY{n+nn}{torch}\PY{n+nn}{.}\PY{n+nn}{nn} \PY{k}{as} \PY{n+nn}{nn}
        \PY{k+kn}{from} \PY{n+nn}{torch}\PY{n+nn}{.}\PY{n+nn}{autograd} \PY{k}{import} \PY{n}{Variable}
        \PY{k+kn}{from} \PY{n+nn}{torchvision} \PY{k}{import} \PY{n}{models}
        \PY{k+kn}{from} \PY{n+nn}{torchvision} \PY{k}{import} \PY{n}{transforms}\PY{p}{,} \PY{n}{utils}
        \PY{k+kn}{import} \PY{n+nn}{numpy} \PY{k}{as} \PY{n+nn}{np}
        \PY{k+kn}{import} \PY{n+nn}{matplotlib}\PY{n+nn}{.}\PY{n+nn}{pyplot} \PY{k}{as} \PY{n+nn}{plt}
        \PY{o}{\PYZpc{}}\PY{k}{matplotlib} inline
        \PY{k+kn}{from} \PY{n+nn}{PIL} \PY{k}{import} \PY{n}{Image}\PY{p}{,} \PY{n}{ImageFilter}\PY{p}{,} \PY{n}{ImageChops}
        
        \PY{n}{GPU\PYZus{}PRESENT} \PY{o}{=} \PY{k+kc}{False} \PY{c+c1}{\PYZsh{}Set to True if you have GPU on your machine}
\end{Verbatim}


    \section{DeepDream}\label{deepdream}

DeepDream is an algorithm that uses CNNs to generate psychedelic-looking
images. It allows us to visualize what features a particular layer of
pretrained CNNs have learned.

\begin{figure}
\centering
\includegraphics{./features.png}
\caption{img}
\end{figure}

First, a base image is fed to the pretrained CNN and forward pass is
done until a particular layer. In order to visualize what that
particular layer has learned, we need to maximize the activations
through that layer. The gradients of that layer are set equal to the
activations from that layer, and then gradient ascent is done on the
input image (gradient ascent is similar to gradient descent except that
instead of minimizing the cost, we are trying to maximize the cost).
This maximizes the activations of that layer.

However, doing just this much does not produce good images. Various
techniques are used to make the resulting image better. Gaussian
blurring can be done to make the image smoother.

One main concept in making images better is the use of octaves. Input
image is repeatedly downscaled, and gradient ascent is applied to all
the images, and then the result is merged into a single output image.

To learn more, read this
\href{https://ai.googleblog.com/2015/06/inceptionism-going-deeper-into-neural.html}{blog
post} from Google.

\begin{figure}
\centering
\includegraphics{./explanation.png}
\caption{img}
\end{figure}

    Here are some helper functions that help you preprocess the image. You
do not need to read these.

    \begin{Verbatim}[commandchars=\\\{\}]
{\color{incolor}In [{\color{incolor}2}]:} \PY{k}{def} \PY{n+nf}{load\PYZus{}image}\PY{p}{(}\PY{n}{path}\PY{p}{)}\PY{p}{:}
            \PY{n}{image} \PY{o}{=} \PY{n}{Image}\PY{o}{.}\PY{n}{open}\PY{p}{(}\PY{n}{path}\PY{p}{)}
            \PY{n}{plt}\PY{o}{.}\PY{n}{figure}\PY{p}{(}\PY{n}{figsize}\PY{o}{=}\PY{p}{(}\PY{l+m+mi}{15}\PY{p}{,}\PY{l+m+mi}{15}\PY{p}{)}\PY{p}{)}
            \PY{n}{plt}\PY{o}{.}\PY{n}{imshow}\PY{p}{(}\PY{n}{image}\PY{p}{)}
            \PY{n}{plt}\PY{o}{.}\PY{n}{title}\PY{p}{(}\PY{l+s+s2}{\PYZdq{}}\PY{l+s+s2}{Base Image}\PY{l+s+s2}{\PYZdq{}}\PY{p}{)}
            \PY{k}{return} \PY{n}{image}
        
        \PY{n}{normalise} \PY{o}{=} \PY{n}{transforms}\PY{o}{.}\PY{n}{Normalize}\PY{p}{(}
            \PY{n}{mean}\PY{o}{=}\PY{p}{[}\PY{l+m+mf}{0.485}\PY{p}{,} \PY{l+m+mf}{0.456}\PY{p}{,} \PY{l+m+mf}{0.406}\PY{p}{]}\PY{p}{,}
            \PY{n}{std}\PY{o}{=}\PY{p}{[}\PY{l+m+mf}{0.229}\PY{p}{,} \PY{l+m+mf}{0.224}\PY{p}{,} \PY{l+m+mf}{0.225}\PY{p}{]}
            \PY{p}{)}
        
        \PY{n}{preprocess} \PY{o}{=} \PY{n}{transforms}\PY{o}{.}\PY{n}{Compose}\PY{p}{(}\PY{p}{[}
            \PY{n}{transforms}\PY{o}{.}\PY{n}{Resize}\PY{p}{(}\PY{p}{(}\PY{l+m+mi}{224}\PY{p}{,}\PY{l+m+mi}{224}\PY{p}{)}\PY{p}{)}\PY{p}{,}
            \PY{n}{transforms}\PY{o}{.}\PY{n}{ToTensor}\PY{p}{(}\PY{p}{)}\PY{p}{,}
            \PY{n}{normalise}
            \PY{p}{]}\PY{p}{)}
        
        \PY{k}{def} \PY{n+nf}{deprocess}\PY{p}{(}\PY{n}{image}\PY{p}{)}\PY{p}{:}
            \PY{k}{if} \PY{n}{GPU\PYZus{}PRESENT}\PY{p}{:}
                \PY{k}{return} \PY{n}{image} \PY{o}{*} \PY{n}{torch}\PY{o}{.}\PY{n}{Tensor}\PY{p}{(}\PY{p}{[}\PY{l+m+mf}{0.229}\PY{p}{,} \PY{l+m+mf}{0.224}\PY{p}{,} \PY{l+m+mf}{0.225}\PY{p}{]}\PY{p}{)}\PY{o}{.}\PY{n}{cuda}\PY{p}{(}\PY{p}{)}  \PY{o}{+} \PY{n}{torch}\PY{o}{.}\PY{n}{Tensor}\PY{p}{(}\PY{p}{[}\PY{l+m+mf}{0.485}\PY{p}{,} \PY{l+m+mf}{0.456}\PY{p}{,} \PY{l+m+mf}{0.406}\PY{p}{]}\PY{p}{)}\PY{o}{.}\PY{n}{cuda}\PY{p}{(}\PY{p}{)}
            \PY{k}{return} \PY{n}{image} \PY{o}{*} \PY{n}{torch}\PY{o}{.}\PY{n}{Tensor}\PY{p}{(}\PY{p}{[}\PY{l+m+mf}{0.229}\PY{p}{,} \PY{l+m+mf}{0.224}\PY{p}{,} \PY{l+m+mf}{0.225}\PY{p}{]}\PY{p}{)}  \PY{o}{+} \PY{n}{torch}\PY{o}{.}\PY{n}{Tensor}\PY{p}{(}\PY{p}{[}\PY{l+m+mf}{0.485}\PY{p}{,} \PY{l+m+mf}{0.456}\PY{p}{,} \PY{l+m+mf}{0.406}\PY{p}{]}\PY{p}{)}
\end{Verbatim}


    We will be using a pretrained CNN called VGG16. You are free to explore
other CNNs too.

    \begin{Verbatim}[commandchars=\\\{\}]
{\color{incolor}In [{\color{incolor}3}]:} \PY{n}{vgg} \PY{o}{=} \PY{n}{models}\PY{o}{.}\PY{n}{vgg16}\PY{p}{(}\PY{n}{pretrained}\PY{o}{=}\PY{k+kc}{True}\PY{p}{)}
        \PY{k}{if} \PY{n}{GPU\PYZus{}PRESENT}\PY{p}{:}
            \PY{n}{vgg} \PY{o}{=} \PY{n}{vgg}\PY{o}{.}\PY{n}{cuda}\PY{p}{(}\PY{p}{)}
        \PY{n+nb}{print}\PY{p}{(}\PY{n}{vgg}\PY{p}{)}
        \PY{n}{modulelist} \PY{o}{=} \PY{n+nb}{list}\PY{p}{(}\PY{n}{vgg}\PY{o}{.}\PY{n}{features}\PY{o}{.}\PY{n}{modules}\PY{p}{(}\PY{p}{)}\PY{p}{)}
\end{Verbatim}


    \begin{Verbatim}[commandchars=\\\{\}]
Downloading: "https://download.pytorch.org/models/vgg16-397923af.pth" to /Users/gautam/.torch/models/vgg16-397923af.pth
100\%|██████████| 553433881/553433881 [01:36<00:00, 5737217.08it/s] 

    \end{Verbatim}

    \begin{Verbatim}[commandchars=\\\{\}]
VGG(
  (features): Sequential(
    (0): Conv2d(3, 64, kernel\_size=(3, 3), stride=(1, 1), padding=(1, 1))
    (1): ReLU(inplace)
    (2): Conv2d(64, 64, kernel\_size=(3, 3), stride=(1, 1), padding=(1, 1))
    (3): ReLU(inplace)
    (4): MaxPool2d(kernel\_size=2, stride=2, padding=0, dilation=1, ceil\_mode=False)
    (5): Conv2d(64, 128, kernel\_size=(3, 3), stride=(1, 1), padding=(1, 1))
    (6): ReLU(inplace)
    (7): Conv2d(128, 128, kernel\_size=(3, 3), stride=(1, 1), padding=(1, 1))
    (8): ReLU(inplace)
    (9): MaxPool2d(kernel\_size=2, stride=2, padding=0, dilation=1, ceil\_mode=False)
    (10): Conv2d(128, 256, kernel\_size=(3, 3), stride=(1, 1), padding=(1, 1))
    (11): ReLU(inplace)
    (12): Conv2d(256, 256, kernel\_size=(3, 3), stride=(1, 1), padding=(1, 1))
    (13): ReLU(inplace)
    (14): Conv2d(256, 256, kernel\_size=(3, 3), stride=(1, 1), padding=(1, 1))
    (15): ReLU(inplace)
    (16): MaxPool2d(kernel\_size=2, stride=2, padding=0, dilation=1, ceil\_mode=False)
    (17): Conv2d(256, 512, kernel\_size=(3, 3), stride=(1, 1), padding=(1, 1))
    (18): ReLU(inplace)
    (19): Conv2d(512, 512, kernel\_size=(3, 3), stride=(1, 1), padding=(1, 1))
    (20): ReLU(inplace)
    (21): Conv2d(512, 512, kernel\_size=(3, 3), stride=(1, 1), padding=(1, 1))
    (22): ReLU(inplace)
    (23): MaxPool2d(kernel\_size=2, stride=2, padding=0, dilation=1, ceil\_mode=False)
    (24): Conv2d(512, 512, kernel\_size=(3, 3), stride=(1, 1), padding=(1, 1))
    (25): ReLU(inplace)
    (26): Conv2d(512, 512, kernel\_size=(3, 3), stride=(1, 1), padding=(1, 1))
    (27): ReLU(inplace)
    (28): Conv2d(512, 512, kernel\_size=(3, 3), stride=(1, 1), padding=(1, 1))
    (29): ReLU(inplace)
    (30): MaxPool2d(kernel\_size=2, stride=2, padding=0, dilation=1, ceil\_mode=False)
  )
  (classifier): Sequential(
    (0): Linear(in\_features=25088, out\_features=4096, bias=True)
    (1): ReLU(inplace)
    (2): Dropout(p=0.5)
    (3): Linear(in\_features=4096, out\_features=4096, bias=True)
    (4): ReLU(inplace)
    (5): Dropout(p=0.5)
    (6): Linear(in\_features=4096, out\_features=1000, bias=True)
  )
)

    \end{Verbatim}

    \subsection{Main functions}\label{main-functions}

\subsubsection{dreamify}\label{dreamify}

This is the actual deep\_dream code. It takes an input image, makes a
forward pass till a particular layer, and then updates the input image
by gradient ascent.

\subsubsection{deep\_dream\_vgg}\label{deep_dream_vgg}

This is a recursive function. It repeatedly downscales the image, then
calls dreamify. Then it upscales the result, and merges (blends) it to
the image at one level higher on the recursive tree. The final image is
the same size as the input image.

    \begin{Verbatim}[commandchars=\\\{\}]
{\color{incolor}In [{\color{incolor}4}]:} \PY{k}{def} \PY{n+nf}{dreamify}\PY{p}{(}\PY{n}{image}\PY{p}{,} \PY{n}{layer}\PY{p}{,} \PY{n}{iterations}\PY{p}{,} \PY{n}{lr}\PY{p}{)}\PY{p}{:}        
            \PY{k}{if} \PY{n}{GPU\PYZus{}PRESENT}\PY{p}{:}
                \PY{n+nb}{input} \PY{o}{=} \PY{n}{Variable}\PY{p}{(}\PY{n}{preprocess}\PY{p}{(}\PY{n}{image}\PY{p}{)}\PY{o}{.}\PY{n}{unsqueeze}\PY{p}{(}\PY{l+m+mi}{0}\PY{p}{)}\PY{o}{.}\PY{n}{cuda}\PY{p}{(}\PY{p}{)}\PY{p}{,} \PY{n}{requires\PYZus{}grad}\PY{o}{=}\PY{k+kc}{True}\PY{p}{)}
            \PY{k}{else}\PY{p}{:}
                \PY{n+nb}{input} \PY{o}{=} \PY{n}{Variable}\PY{p}{(}\PY{n}{preprocess}\PY{p}{(}\PY{n}{image}\PY{p}{)}\PY{o}{.}\PY{n}{unsqueeze}\PY{p}{(}\PY{l+m+mi}{0}\PY{p}{)}\PY{p}{,} \PY{n}{requires\PYZus{}grad}\PY{o}{=}\PY{k+kc}{True}\PY{p}{)}
            \PY{n}{vgg}\PY{o}{.}\PY{n}{zero\PYZus{}grad}\PY{p}{(}\PY{p}{)}
            \PY{k}{for} \PY{n}{i} \PY{o+ow}{in} \PY{n+nb}{range}\PY{p}{(}\PY{n}{iterations}\PY{p}{)}\PY{p}{:}
        
                \PY{n}{out} \PY{o}{=} \PY{n+nb}{input}
                \PY{k}{for} \PY{n}{j} \PY{o+ow}{in} \PY{n+nb}{range}\PY{p}{(}\PY{n}{layer}\PY{p}{)}\PY{p}{:}
                    \PY{n}{out} \PY{o}{=} \PY{n}{modulelist}\PY{p}{[}\PY{n}{j}\PY{o}{+}\PY{l+m+mi}{1}\PY{p}{]}\PY{p}{(}\PY{n}{out}\PY{p}{)}
                \PY{n}{loss} \PY{o}{=} \PY{n}{out}\PY{o}{.}\PY{n}{norm}\PY{p}{(}\PY{p}{)}
                \PY{n}{loss}\PY{o}{.}\PY{n}{backward}\PY{p}{(}\PY{p}{)}
                \PY{n+nb}{input}\PY{o}{.}\PY{n}{data} \PY{o}{=} \PY{n+nb}{input}\PY{o}{.}\PY{n}{data} \PY{o}{+} \PY{n}{lr} \PY{o}{*} \PY{n+nb}{input}\PY{o}{.}\PY{n}{grad}\PY{o}{.}\PY{n}{data}
            
            \PY{n+nb}{input} \PY{o}{=} \PY{n+nb}{input}\PY{o}{.}\PY{n}{data}\PY{o}{.}\PY{n}{squeeze}\PY{p}{(}\PY{p}{)}
            \PY{n+nb}{input}\PY{o}{.}\PY{n}{transpose\PYZus{}}\PY{p}{(}\PY{l+m+mi}{0}\PY{p}{,}\PY{l+m+mi}{1}\PY{p}{)}
            \PY{n+nb}{input}\PY{o}{.}\PY{n}{transpose\PYZus{}}\PY{p}{(}\PY{l+m+mi}{1}\PY{p}{,}\PY{l+m+mi}{2}\PY{p}{)}
            \PY{n+nb}{input} \PY{o}{=} \PY{n}{np}\PY{o}{.}\PY{n}{clip}\PY{p}{(}\PY{n}{deprocess}\PY{p}{(}\PY{n+nb}{input}\PY{p}{)}\PY{p}{,} \PY{l+m+mi}{0}\PY{p}{,} \PY{l+m+mi}{1}\PY{p}{)}
            \PY{n}{im} \PY{o}{=} \PY{n}{Image}\PY{o}{.}\PY{n}{fromarray}\PY{p}{(}\PY{n}{np}\PY{o}{.}\PY{n}{uint8}\PY{p}{(}\PY{n+nb}{input}\PY{o}{*}\PY{l+m+mi}{255}\PY{p}{)}\PY{p}{)}
            \PY{k}{return} \PY{n}{im}
\end{Verbatim}


    \begin{Verbatim}[commandchars=\\\{\}]
{\color{incolor}In [{\color{incolor}5}]:} \PY{k}{def} \PY{n+nf}{deep\PYZus{}dream\PYZus{}vgg}\PY{p}{(}\PY{n}{image}\PY{p}{,} \PY{n}{layer}\PY{p}{,} \PY{n}{iterations}\PY{p}{,} \PY{n}{lr}\PY{p}{,} \PY{n}{octave\PYZus{}scale}\PY{p}{,} \PY{n}{num\PYZus{}octaves}\PY{p}{)}\PY{p}{:}
            
            \PY{k}{if} \PY{n}{num\PYZus{}octaves} \PY{o}{\PYZgt{}} \PY{l+m+mi}{0}\PY{p}{:}
                \PY{n}{image1} \PY{o}{=} \PY{n}{image}\PY{o}{.}\PY{n}{filter}\PY{p}{(}\PY{n}{ImageFilter}\PY{o}{.}\PY{n}{GaussianBlur}\PY{p}{(}\PY{l+m+mi}{2}\PY{p}{)}\PY{p}{)}
                \PY{k}{if}\PY{p}{(}\PY{n}{image1}\PY{o}{.}\PY{n}{size}\PY{p}{[}\PY{l+m+mi}{0}\PY{p}{]}\PY{o}{/}\PY{n}{octave\PYZus{}scale} \PY{o}{\PYZlt{}} \PY{l+m+mi}{1} \PY{o+ow}{or} \PY{n}{image1}\PY{o}{.}\PY{n}{size}\PY{p}{[}\PY{l+m+mi}{1}\PY{p}{]}\PY{o}{/}\PY{n}{octave\PYZus{}scale}\PY{o}{\PYZlt{}}\PY{l+m+mi}{1}\PY{p}{)}\PY{p}{:}
                    \PY{n}{size} \PY{o}{=} \PY{n}{image1}\PY{o}{.}\PY{n}{size}
                \PY{k}{else}\PY{p}{:}
                    \PY{n}{size} \PY{o}{=} \PY{p}{(}\PY{n+nb}{int}\PY{p}{(}\PY{n}{image1}\PY{o}{.}\PY{n}{size}\PY{p}{[}\PY{l+m+mi}{0}\PY{p}{]}\PY{o}{/}\PY{n}{octave\PYZus{}scale}\PY{p}{)}\PY{p}{,} \PY{n+nb}{int}\PY{p}{(}\PY{n}{image1}\PY{o}{.}\PY{n}{size}\PY{p}{[}\PY{l+m+mi}{1}\PY{p}{]}\PY{o}{/}\PY{n}{octave\PYZus{}scale}\PY{p}{)}\PY{p}{)}
                    
                \PY{n}{image1} \PY{o}{=} \PY{n}{image1}\PY{o}{.}\PY{n}{resize}\PY{p}{(}\PY{n}{size}\PY{p}{,}\PY{n}{Image}\PY{o}{.}\PY{n}{ANTIALIAS}\PY{p}{)}
                \PY{n}{image1} \PY{o}{=} \PY{n}{deep\PYZus{}dream\PYZus{}vgg}\PY{p}{(}\PY{n}{image}\PY{p}{,} \PY{n}{layer}\PY{p}{,} \PY{n}{iterations}\PY{p}{,} \PY{n}{lr}\PY{p}{,} \PY{n}{octave\PYZus{}scale}\PY{p}{,} \PY{n}{num\PYZus{}octaves}\PY{o}{\PYZhy{}}\PY{l+m+mi}{1}\PY{p}{)} \PY{c+c1}{\PYZsh{}TODO: recursively call itself with num\PYZus{}octaves \PYZhy{} 1}
                \PY{n}{size} \PY{o}{=} \PY{p}{(}\PY{n}{image}\PY{o}{.}\PY{n}{size}\PY{p}{[}\PY{l+m+mi}{0}\PY{p}{]}\PY{p}{,} \PY{n}{image}\PY{o}{.}\PY{n}{size}\PY{p}{[}\PY{l+m+mi}{1}\PY{p}{]}\PY{p}{)}
                \PY{n}{image1} \PY{o}{=} \PY{n}{image1}\PY{o}{.}\PY{n}{resize}\PY{p}{(}\PY{n}{size}\PY{p}{,}\PY{n}{Image}\PY{o}{.}\PY{n}{ANTIALIAS}\PY{p}{)}
                \PY{n}{image} \PY{o}{=} \PY{n}{ImageChops}\PY{o}{.}\PY{n}{blend}\PY{p}{(}\PY{n}{image}\PY{p}{,} \PY{n}{image1}\PY{p}{,} \PY{l+m+mf}{0.6}\PY{p}{)}
        
            \PY{n}{img\PYZus{}result} \PY{o}{=} \PY{n}{dreamify}\PY{p}{(}\PY{n}{image}\PY{p}{,} \PY{n}{layer}\PY{p}{,} \PY{n}{iterations}\PY{p}{,} \PY{n}{lr}\PY{p}{)}
            \PY{n}{img\PYZus{}result} \PY{o}{=} \PY{n}{img\PYZus{}result}\PY{o}{.}\PY{n}{resize}\PY{p}{(}\PY{n}{image}\PY{o}{.}\PY{n}{size}\PY{p}{)}
            \PY{k}{return} \PY{n}{img\PYZus{}result}
\end{Verbatim}


    Load a base image. Feel free to load your own image.

    \begin{Verbatim}[commandchars=\\\{\}]
{\color{incolor}In [{\color{incolor}6}]:} \PY{n}{sky} \PY{o}{=} \PY{n}{load\PYZus{}image}\PY{p}{(}\PY{l+s+s1}{\PYZsq{}}\PY{l+s+s1}{sather.png}\PY{l+s+s1}{\PYZsq{}}\PY{p}{)}
\end{Verbatim}


    \begin{center}
    \adjustimage{max size={0.9\linewidth}{0.9\paperheight}}{output_10_0.png}
    \end{center}
    { \hspace*{\fill} \\}
    
    Notice that the shallow layers learn basic shapes, lines, edges. After
that, layers learn patterns. And the deeper layers learn much more
complex features like eyes, face, etc.

    \begin{Verbatim}[commandchars=\\\{\}]
{\color{incolor}In [{\color{incolor}7}]:} \PY{n}{sky\PYZus{}5} \PY{o}{=} \PY{n}{deep\PYZus{}dream\PYZus{}vgg}\PY{p}{(}\PY{n}{sky}\PY{p}{,} \PY{l+m+mi}{5}\PY{p}{,} \PY{l+m+mi}{5}\PY{p}{,} \PY{l+m+mf}{0.3}\PY{p}{,} \PY{l+m+mi}{2}\PY{p}{,} \PY{l+m+mi}{20}\PY{p}{)}
        \PY{n}{plt}\PY{o}{.}\PY{n}{figure}\PY{p}{(}\PY{n}{figsize}\PY{o}{=}\PY{p}{(}\PY{l+m+mi}{15}\PY{p}{,} \PY{l+m+mi}{15}\PY{p}{)}\PY{p}{)}
        \PY{n}{plt}\PY{o}{.}\PY{n}{imshow}\PY{p}{(}\PY{n}{sky\PYZus{}5}\PY{p}{)}
\end{Verbatim}


\begin{Verbatim}[commandchars=\\\{\}]
{\color{outcolor}Out[{\color{outcolor}7}]:} <matplotlib.image.AxesImage at 0x1476fbb38>
\end{Verbatim}
            
    \begin{center}
    \adjustimage{max size={0.9\linewidth}{0.9\paperheight}}{output_12_1.png}
    \end{center}
    { \hspace*{\fill} \\}
    
    \begin{Verbatim}[commandchars=\\\{\}]
{\color{incolor}In [{\color{incolor}8}]:} \PY{n}{sky\PYZus{}17} \PY{o}{=} \PY{n}{deep\PYZus{}dream\PYZus{}vgg}\PY{p}{(}\PY{n}{sky}\PY{p}{,} \PY{l+m+mi}{17}\PY{p}{,} \PY{l+m+mi}{3}\PY{p}{,} \PY{l+m+mf}{0.3}\PY{p}{,} \PY{l+m+mi}{2}\PY{p}{,} \PY{l+m+mi}{20}\PY{p}{)}
        \PY{n}{plt}\PY{o}{.}\PY{n}{figure}\PY{p}{(}\PY{n}{figsize}\PY{o}{=}\PY{p}{(}\PY{l+m+mi}{15}\PY{p}{,} \PY{l+m+mi}{15}\PY{p}{)}\PY{p}{)}
        \PY{n}{plt}\PY{o}{.}\PY{n}{imshow}\PY{p}{(}\PY{n}{sky\PYZus{}17}\PY{p}{)}
\end{Verbatim}


\begin{Verbatim}[commandchars=\\\{\}]
{\color{outcolor}Out[{\color{outcolor}8}]:} <matplotlib.image.AxesImage at 0x1476788d0>
\end{Verbatim}
            
    \begin{center}
    \adjustimage{max size={0.9\linewidth}{0.9\paperheight}}{output_13_1.png}
    \end{center}
    { \hspace*{\fill} \\}
    
    \begin{Verbatim}[commandchars=\\\{\}]
{\color{incolor}In [{\color{incolor}9}]:} \PY{n}{sky\PYZus{}26} \PY{o}{=} \PY{n}{deep\PYZus{}dream\PYZus{}vgg}\PY{p}{(}\PY{n}{sky}\PY{p}{,} \PY{l+m+mi}{26}\PY{p}{,} \PY{l+m+mi}{5}\PY{p}{,} \PY{l+m+mf}{0.2}\PY{p}{,} \PY{l+m+mi}{2}\PY{p}{,} \PY{l+m+mi}{20}\PY{p}{)}
        \PY{n}{plt}\PY{o}{.}\PY{n}{figure}\PY{p}{(}\PY{n}{figsize}\PY{o}{=}\PY{p}{(}\PY{l+m+mi}{15}\PY{p}{,} \PY{l+m+mi}{15}\PY{p}{)}\PY{p}{)}
        \PY{n}{plt}\PY{o}{.}\PY{n}{imshow}\PY{p}{(}\PY{n}{sky\PYZus{}26}\PY{p}{)}
\end{Verbatim}


\begin{Verbatim}[commandchars=\\\{\}]
{\color{outcolor}Out[{\color{outcolor}9}]:} <matplotlib.image.AxesImage at 0x1420ee240>
\end{Verbatim}
            
    \begin{center}
    \adjustimage{max size={0.9\linewidth}{0.9\paperheight}}{output_14_1.png}
    \end{center}
    { \hspace*{\fill} \\}
    

    % Add a bibliography block to the postdoc
    
    
    
    \end{document}
